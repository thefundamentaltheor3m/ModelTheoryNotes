\section{Cardinality and Categoricity}

Throughout this section, let $\Lang$ be a language.

\begin{boxdefinition}[Cardinality of a Structure]
    Let $M$ be an $\Lang$-structure. The \textbf{cardinality of $M$}, denoted $\norm{M}$, is the cardinality of its universe $\abs{M}$.
\end{boxdefinition}

We can also talk about the size of a theory.

\begin{boxdefinition}[Categoricity of a Theory]\label{Ch1:Def:Categoricity}
    Let $T$ be an $\Lang$-theory. Suppose $\lambda \geq \abs{L}$ is a cardinal. We say that \textbf{$T$ is $\lambda$-Categorical}, or that \textbf{$T$ is categorical in $\lambda$}, if for all $\Lang$-structures $M$ and $N$ such that $M, N \models T$ and $\norm{M} = \norm{N} = \lambda$, we have that $M \cong N$.
\end{boxdefinition}

Categoricity brings up interesting questions, such as the so-called \textit{spectrum problem}.

\subsection{The Spectrum Problem}

The spectrum of a theory with respect to a cardinal is defined as follows.

\begin{boxdefinition}[Spectrum]\label{Ch1:Def:Spectrum}
    Let $T$ be an $\Lang$-theory and let $\lambda$ be a cardinal. We define the \textbf{spectrum of $T$ with respect to $\lambda$} to be
    \begin{align*}
        I\of{\lambda, T} := \abs{
            \setst{\quotient{M}{\cong}}{M \models T \text{ and } \norm{M} = \lambda}
        }
    \end{align*}
    ie, $I\of{\lambda, T}$ denotes the number of isomorphism classes of models of $T$ of cardinality $\lambda$.
\end{boxdefinition}

It is obvious, from \Cref{Ch1:Def:Categoricity}, that a theory $T$ is $\lambda$-categorical if and only if $I\of{\lambda, T} = 1$. However, if $T$ is not $\lambda$-categorical, then it is, in general, quite difficult to compute $I\of{\lambda, T}$. In fact, for most theories and cardinals, computing the spectrum is an \textit{open problem}, referred to as the \textbf{spectrum problem}.

There has been some progress on this problem. Steinitz made the following determinations.

\begin{boxtheorem}[Steinitz]
    Let $\Lang$ be the language of fields, and let $T$ be the theory of algebraically closed fields of characteristic $p$ (obtained by adding the appropriate sentences to the Theory of Fields encountered in \Cref{Ch1:Eg:Thy_of_Fields}). Then,
    \begin{enumerate}[label = \normalfont \arabic*.]
        \item $I\of{\aleph_{0}, T} = \aleph_{0}$.
        \item For all $\lambda > \aleph_{0}$, $I\of{\lambda, T} = 1$.
    \end{enumerate}
\end{boxtheorem}

The spectrum problem has been worked on by some of the most eminent logicians of our time, including Rami's advisor, Saharon Shelah, who proved a famous conjecture by Morley (1965). More on the Spectrum Problem can be found on the associated \href{https://en.wikipedia.org/wiki/Spectrum_of_a_theory#:~:text=More%20precisely%2C%20for%20any%20complete,of%20a%20countable%20theory%20T.}{Wikipedia page}, and while this is not the most authoritative source, its contents are nonetheless interesting.

Morley also proved a famous conjecture by Łos from the 1950s, which since became known as Morley's Categoricity Theorem.

\begin{boxtheorem}[Morley's Categoricity Theorem, Morley 1965]
    Let $T$ be a theory in a language $\Lang$. Assume that $\abs{\Lang} \leq \aleph_{0}$. If $\exists \lambda > \aleph_{0}$ such that $T$ is $\lambda$-categorical, them $\forall \lambda > \aleph_{0}$, $T$ is $\lambda$-categorical.
\end{boxtheorem}

One of our objectives in this course is to prove Morley's Categoricity Theorem.

As a side note, Morley was initially a PhD student of Saunders MacLane's at the University of Chicago. Morley didn't initially finish his PhD, to the point of losing his stipend at Chicago, but somehow landed a job at Berkeley, where he proved this famous theorem. MacLane, a staunch category theorist, didn't believe Morley's work was quite enough to merit a PhD; nevertheless, after being persuaded by the then-nascent (and very excited) model theory community, he eventually relented and awarded Morley his degree.

Here, we end our discussion on the spectrum problem. Before proceeding further, we recall the basics of cardinal arithmetic.

\subsection{Cardinal Arithmetic}

We begin by introducing notation.

\begin{boxnotation}
    We denote by
    \begin{itemize}
        \item $\ZF$ the \ZFA\ of Set Theory
        \item $\AC$ the \ACA
        \item $\ZFC$ the \ZFCA
    \end{itemize}
\end{boxnotation}

We denote cardinality of a set $A$ by $\abs{A}$ or $\card{A}$ and write $\abs{A} = \abs{B}$ if and only if there is a bijection from $A$ to $B$. Informally, a \textbf{cardinal} is a measure of cardinality. That is, a set $\lambda$ is a cardinal if $\lambda = \abs{A}$ for some set $A$. We denote by $\aleph_{0}$ the cardinal of the natural numbers, which we will denote $\omega$ in any cardinal- or ordinal-theoretic context.

There are more precise ways in which we can define the notions of ordinals and cardinals. We do not do this here, but we mention that there is an appendix in Rami's book and several sections in my undergrad logic lecture notes\todo{Add references} that discuss this.

\begin{boxdefinition}[Cardinal Arithmetic]
    Let $\lambda, \mu$ be cardinals, with $\lambda = \abs{A}$ and $\mu = \abs{B}$. We denote
    \begin{align*}
        \lambda + \mu &:= \sorry \\
        \lambda \cdot \mu &:= \abs{A \times B}
    \end{align*}
\end{boxdefinition}

The following is a famed theorem of Tarski, a direct consequence of which is precisely the fundamental theorem of cardinal arithmetic.

\begin{boxtheorem}[Tarski]
    We can make the following deduction:
    \begin{align*}
        \ZF \vdash \parenth{\AC \lr \forall A, \, \abs{A} \geq \aleph_{0} \to \abs{A \times A} = \abs{A}}
    \end{align*}
    Equivalently,
    \begin{align*}
        \ZF \vdash \parenth{\AC \lr \forall A, \, \lambda \geq \aleph_{0} \to \lambda \cdot \lambda = \lambda}
    \end{align*}
\end{boxtheorem}

The fundamental theorem of cardinal arithmetic, which states that $\abs{\omega \times \omega} = \abs{\omega}$, is clearly just the specialisation of the above result to the case where $\lambda = \aleph_{0}$.

There is another fact that will be important for our purposes.

\begin{boxtheorem}
    For infinite cardinals $\lambda, \mu \geq \aleph_{0}$, we have
    \begin{align*}
        \lambda \cdot \mu = \max\of{\lambda, \mu} = \lambda + \mu
    \end{align*}
\end{boxtheorem}

The reason for discussing cardinal arithmetic is that we can exploit it to prove the existence of submodels of specific cardinalities.

\section{A Word on Submodels}

Fix a language $\Lang$.

\subsection{Submodel Existence}

We begin by defining the cardinality of a structure.

\begin{boxdefinition}[Cardinality of a Structure]
    Let $N$ be a $\Lang$-structure. We define $\card{N}$ to be the cardinality of the union of 
\end{boxdefinition}

We begin with the famed submodel theorem.

\begin{boxtheorem}[The Submodel Theorem {\cite[Theorem 2.8, pp. 49-50]{MOAB}}]
    Let $M$ be a $\Lang$-structure. Define $\lambda := \abs{\Lang} + \aleph_{0}$. If $A \leq \abs{M}$, then there exists a substructure $N \leq M$ such that
    \begin{enumerate}[label = (\alph*)]
        \item $\card{N} \geq A$
        \item $\card{\abs{N}} \leq \abs{A} + \lambda$
    \end{enumerate}
\end{boxtheorem}
\begin{proof}
    By recursion on $n < \omega$, define sets $\setst{B_{n} \ssq \abs{M}}{n < \omega}$ such that
    \begin{enumerate}
        \item $B_{0} = \setst{c \in \Consts^M}{c \text{ is a constant symbol of } c} \cup A$
        \item If $n < \omega$, then $\abs{B_n} \leq \abs{A} + \lambda$
        \item For all $n < \omega$, define $B_{n+1} := \setst{F^M\of{\overline{a}}}{\overline{a} \in B_{n}} \cup B_n$
    \end{enumerate}
    This is enough: if we have such a sequence of $B_n$, then we could take $B := \bigcup_{n < \omega} B_n$ and define $N := \cycl{B, F^M, R^M, C^M}$. We can show that this satisfies the desired conditions.
    \begin{enumerate}[label = (\alph*)]
        \item \sorry
        \item \sorry
    \end{enumerate}

    Given these, all that remains now is to show that this is possible. \sorry\todo{Finish using textbook proof}
\end{proof}

\subsection{Elementary Submodels}

Recall the definition of elementary substructures (\sorry). In this subsection, we define an analogous notion for models.

\begin{boxdefinition}[Elementary Submodels]
    Let $M, N$ be $\Lang$-structures. We say that \textbf{$M$ is an elementary submodel of $N$}, denoted $M \preceq N$, if
    \begin{enumerate}
        \item $M \subseteq N$
        \item $M \models \varphi\!\brac{a_1, \ldots, a_n}$ iff $N \models \varphi\!\brac{a_1, \ldots, a_n}$ for every $\varphi \in \Fml\of{\Lang}$ and $a_1, \ldots, a_n \in \abs{M}$.
    \end{enumerate}
\end{boxdefinition}

We can relate this to the notion of elementary substructures in the following manner.

\begin{boxtheorem}[Tarski-Vaught 1956]
    Let $M, N$ be $\Lang$-structures with $M \subseteq N$. If $M \preceq N$, then $M \equiv N$.
\end{boxtheorem}

The converse is not true.

\begin{boxcexample}
    \sorry
    % If $M \equiv N$, then it is not necessary that $M \preceq N$.
\end{boxcexample}

\subsection{The Tarski-Vaught Test}

In this subsection, we explore a monumental result by Tarski and Vaught that gives a sufficient and necessary condition for a substructure to be elementary.

We begin by introducing some notation.

\begin{boxlnotation}
    Denote by $\star_{\psi}$ the statement
    \begin{align}
        \text{For all } a_1, \ldots, a_n \in \abs{M},
        \qquad
        M \models \psi\!\brac{a_1, \ldots, a_n}
        \iff
        N \models \psi\!\brac{a_1, \ldots, a_n}
    \end{align}
    for some $\psi \in \Fml(L)$.
\end{boxlnotation}

Next, we note a fact about substructures.

\begin{boxlemma}\label{Ch1:Lemma:Tarski-Vaught-star-psi}
    Let $N$ be an $L$-structure and let $M \ssq N$ be a substructure of $N$. Then, $\star_{\psi}$ holds for all quantifier-free formulae $\psi \in \Fml(L)$.
\end{boxlemma}
\begin{proof}
    \sorry
\end{proof}

\begin{boxtheorem}[The Tarski-Vaught Test]\label{Ch1:Thm:Tarski-Vaught}
    Let $M, N$ be $L$-structures with $M \ssq N$. Then, the following are equivalent.
    \begin{enumerate}
        \item $M \preceq N$
        \item If, for every $\varphi\of{y, x_1, \ldots, x_n} \in \Fml(L)$ and $a_1, \ldots, a_n \in \abs{M}$,
        \begin{align*}
            N \models \exists y \, \varphi\of{y ,a_1, \ldots, a_n}
        \end{align*}
        then there is some $b \in \abs{M}$ such that $N \models \varphi\!\brac{b, a_1, \ldots, a_n}$
    \end{enumerate}
\end{boxtheorem}
\begin{remark}
    We can see this as 'a more ``algebraic'' notion of being a submodel.' What is a formula? A list of quantifiers, connectives, etc. - for example, we can think of polynomials in several variables, which we wish to solve. If $\vp(y,x)$ is a set of finitely many equations (which we wish to solve), we can see this result as telling us that if there exists a solution to the system $y\in N$, there is also a $b$ in the substructure $M$ which also solves the same system. 
\end{remark}
\begin{proof}[Proof of \Cref{Ch1:Thm:Tarski-Vaught}.]
    \begin{description}
        \item[\underline{$1 \implies 2$.}]
        Fix  $\varphi\of{y, x_1, \ldots, x_n} \in \Fml(L)$ and $a_1, \ldots, a_n \in \abs{M}$. Suppose
        \begin{align*}
            N \models \exists y \, \varphi\of{y ,a_1, \ldots, a_n}
        \end{align*}
        Since $M \preceq N$, by definition of satisfaction, we know that
        \begin{align*}
            M \models \exists y \, \varphi\of{y ,a_1, \ldots, a_n}
        \end{align*}
        This tells us that there is $b \in \abs{M}$ witnessing $\varphi$, meaning that
        \begin{align*}
            M \models \exists y \, \varphi\of{b ,a_1, \ldots, a_n}
        \end{align*}
        Then, since $M \preceq N$, we have that
        \begin{align*}
            N \models \exists y \, \varphi\of{b ,a_1, \ldots, a_n}
        \end{align*}
        as required.

        \item[\underline{$2 \implies 1$.}]
        We show, by induction on $\varphi\of{y, x_1, \ldots, x_n} \in \Fml(L)$, that $\star_{\psi}$ holds for all $\psi \in \Fml(L)$. Recall, from \Cref{Ch1:Lemma:Tarski-Vaught-star-psi}, that $\star_{\psi}$ does hold for quantifier-free formulae $\psi$. In particular, it holds for atomic formulae. We can now consider the different possible cases on $\psi$.
        \begin{enumerate}
            \item \underline{$\psi$ is of the form $\psi_1 \land \psi_2$.} \newline
            \sorry

            \item \underline{$\psi\of{x_1, \ldots, x_n}$ is of the form $\exists y \, \varphi\of{y, x_1, \ldots, x_n}$.} \newline
            Assume that $M \models \psi\!\brac{a_1, \ldots, a_n}$. Then, by assumption, there is some $b \in \abs{M}$ such that $M \models \varphi\!\brac{b, a_1, \ldots, a_n}$. Then, $N \models \varphi\!\brac{b, a_1, \ldots, a_n}$ because $b \in \abs{M} \ssq \abs{N}$. Thus,
            \begin{align*}
                N \models \exists y \, \varphi\of{y, a_1, \ldots, a_n}
            \end{align*}
            ie,
            \begin{align*}
                N \models \psi\!\brac{a_1, \ldots, a_n}
            \end{align*}
        \end{enumerate}
    \end{description}
    By $\star_{\psi}$, $b \in \abs{M}$ % The truth is I'm a bit confused and a little lost - need to think a little about the reasoning here. Should be ok but need to think about it a bit
\end{proof} %...not alone (heh :)) - there's probably time (which will be made) to do just that :D Sounds good

\subsection{Chains of Substructures}

Throughout this subsection, fix a linear order $\parenth{\I, \le}$.

\begin{boxdefinition}[Chain]
    Let $\setst{M_{i}}{i \in \I}$ be $\Lang$-structures. We say they form a \textbf{chain} if for all $i_1, i_2 \in \I$, if $i_1 < i_2$ then $M_{i_1} \subseteq M_{i_2}$.
\end{boxdefinition}

\begin{boxlnotation}
    For the remainder of this subsection, fix a chain $\setst{M_{i}}{i \in \I}$. Define
    \begin{align*}
        N := \bigcup_{i \in \I} M_i
    \end{align*}
\end{boxlnotation}
It is easy to show that $N$ is an $\Lang$-structure as well. Moreover, $M_i \subseteq N$ for all $i \in \I$.

We can ask ourselves a natural question: suppose $T$ is a theory in $\Lang$ and that $\forall i \in \I$, $M_i \models T$. Is it necessarily true that $N \models T$ as well?

The answer turns out to be no when $\abs{I} \geq \aleph_0$.

\begin{boxcexample}
    Take $I = \omega$ and \sorry
\end{boxcexample}

We can instead define elementary chains, which are the analogues of chains for elementary substructures.

%just not the elementary part? yeah you're right (yes, we do have chains)
\begin{boxdefinition}[Elementary Chain] % I got this dw
    We say that $\setst{M_{i}}{i \in \I}$ form an \textbf{elementary chain} if for all $i_1, i_2 \in \I$, if $i_1 < i_2$ then $M_{i_1} \preceq M_{i_2}$.
\end{boxdefinition}

We can apply \Cref{Ch1:Thm:Tarski-Vaught} to prove an important result on elementary chains.

\begin{boxtheorem}[Tarski-Vaught Chain Theorem]
    Assume $\setst{M_i}{i \in I}$ is an elementary chain. Let $N$ be an $L$-structure. Then, the $L$-structure
    \begin{align*}
        N = \bigcup_{i \in I} M_i
    \end{align*}
    satisfies the property that for all $i \in I$, $M_i \preceq N$.
\end{boxtheorem}
\begin{proof}
    Since we already know that $M_i \ssq N$, it is enough to show that for every $\psi\of{x_1, \ldots, x_n} \in \Fml(L)$ and every $i \in I$, $\star_{\psi}$ holds (where $\star_{\psi}$ is the formula defined as local notation in the previous subsection, considered along with the substructure $M_i$ of $N$).

    We know that for all $i \in I$, since $M_i \ssq N$, $\star_{\psi}$ holds for atomic formula. We induct on logical connectives and quantifiers to exhaustively prove that the statement $\forall i \in I, \star_{\psi}$ is true. We only do a few cases explicitly.

    \begin{enumerate}
        \item \underline{$\psi\of{x_1, \ldots, x_n}$ is of the form $\neg \vp\of{x_1, \ldots, x_n}$.}

        Then, for all $i \in I$, $M_i \models \psi\!\brac{a_1, \ldots, a_n}$ iff $M_i \not\models \vp\!\brac{a_1, \ldots, a_n}$. By the induction hypothesis that $\forall i \in I, \star_{\phi}$ holds for all formulae $\phi$ with fewer quantifiers than $\psi$, we can conclude that
        \begin{align*}
            M_i \not\models \vp\!\brac{a_1, \ldots, a_n}
            \iff
            N \not\models \vp\!\brac{a_1, \ldots, a_n}
        \end{align*}
        This tells us that $N \models \psi\!\brac{a_1, \ldots, a_n}$ for every interpretation $a_i$ of $x_i$.

        \item \underline{$\psi\of{x_1, \ldots, x_n}$ is of the form $\exists y\,  \vp\of{y, x_1, \ldots, x_n}$.}

        Fix $i \in I$. Then, $M_i \models \psi\!\brac{a_1, \ldots, a_n} \implies M_i \models \exists y \, \varphi\!\of{y, a_1, \ldots, a_n}$. Then, there is some $b \in \abs{M_i}$ such that $M_i \models \varphi\!\brac{b, a_1, \ldots, a_n}$. This tells us, by the induction hypothesis, that $N \models \varphi\!\brac{b, a_1, \ldots, a_n}$ for some $b \in \abs{M_i} \ssq \abs{N}$. Thus, $N \models \exists y\,  \vp\of{y, x_1, \ldots, x_n}$, meaning $N \models \psi\!\brac{a_1, \ldots, a_n}$ as required.
    \end{enumerate}
    We can argue similarly for other quantifiers and connectives.
\end{proof}

\begin{boxcorollary}
    If $T$ is an $L$-theory, then if $M_i \models T$ for all $i \in I$ then $N \models T$ as well.
\end{boxcorollary}

We don't prove this corollary here.

We finally mention an additional nuance. Suppose $i \in I$ satisfies $a_1, \ldots, a_n \in \abs{M_i}$ and $N \models \psi\!\brac{a_1, \ldots, a_n}$. Say that $\psi$ is of the form $\exists y \, \varphi\of{y, x_1, \ldots, x_n}$. Then, by the definition of satisfaction, we know that there is some $b \in \abs{N}$

\begin{boxdefinition}[directed poset]
    Let $(I, <)$ be a poset, we say that $I$ is ``directed" if $\forall i, j\in I$ there exists $b\in I$ such that $i\leq k, j\leq k$.
\end{boxdefinition}
\begin{remark}
    This theorem can be extended - we don't need a linearly ordered set, possibly a directed poset would suffice? (We won't see this in this course.)
\end{remark}

%this is why elementary submodel is stronger than having the same theory(?) (just getting all the terms straight in my head, sorry again) - alright, no worries (I will continue doing that)
% Hmm - I might have them mixed up asw, need to look at it again. (I want to say yes) - Lmk once you figure it out!
% Other way round - sentences are formulae without free vars - sure

\subsection{The Löwenheim-Skolem Theorems}