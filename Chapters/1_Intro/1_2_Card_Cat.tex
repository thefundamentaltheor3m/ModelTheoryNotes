\section{Cardinality and Categoricity}

Throughout this section, let $\Lang$ be a language.

\begin{boxdefinition}[Cardinality of a Structure]
    Let $M$ be an $\Lang$-structure. The \textbf{cardinality of $M$}, denoted $\norm{M}$, is the cardinality of its universe $\abs{M}$.
\end{boxdefinition}

We can also talk about the size of a theory.

\begin{boxdefinition}[Categoricity of a Theory]\label{Ch1:Def:Categoricity}
    Let $T$ be an $\Lang$-theory. Suppose $\lambda \geq \abs{L}$ is a cardinal. We say that \textbf{$T$ is $\lambda$-Categorical}, or that \textbf{$T$ is categorical in $\lambda$}, if for all $\Lang$-structures $M$ and $N$ such that $M, N \models T$ and $\norm{M} = \norm{N} = \lambda$, we have that $M \cong N$.
\end{boxdefinition}

Categoricity brings up interesting questions, such as the so-called \textit{spectrum problem}.

\subsection{The Spectrum Problem}

The spectrum of a theory with respect to a cardinal is defined as follows.

\begin{boxdefinition}[Spectrum]\label{Ch1:Def:Spectrum}
    Let $T$ be an $\Lang$-theory and let $\lambda$ be a cardinal. We define the \textbf{spectrum of $T$ with respect to $\lambda$} to be
    \begin{align*}
        I\of{\lambda, T} := \abs{
            \setst{\quotient{M}{\cong}}{M \models T \text{ and } \norm{M} = \lambda}
        }
    \end{align*}
    ie, $I\of{\lambda, T}$ denotes the number of isomorphism classes of models of $T$ of cardinality $\lambda$.
\end{boxdefinition}

It is obvious, from \Cref{Ch1:Def:Categoricity}, that a theory $T$ is $\lambda$-categorical if and only if $I\of{\lambda, T} = 1$. However, if $T$ is not $\lambda$-categorical, then it is, in general, quite difficult to compute $I\of{\lambda, T}$. In fact, for most theories and cardinals, computing the spectrum is an \textit{open problem}, referred to as the \textbf{spectrum problem}.

There has been some progress on this problem. Steinitz made the following determinations.

\begin{boxtheorem}[Steinitz]
    Let $\Lang$ be the language of fields, and let $T$ be the theory of algebraically closed fields of characteristic $p$ (obtained by adding the appropriate sentences to the Theory of Fields encountered in \Cref{Ch1:Eg:Thy_of_Fields}). Then,
    \begin{enumerate}[label = \normalfont \arabic*.]
        \item $I\of{\aleph_{0}, T} = \aleph_{0}$.
        \item For all $\lambda > \aleph_{0}$, $I\of{\lambda, T} = 1$.
    \end{enumerate}
\end{boxtheorem}

The spectrum problem has been worked on by some of the most eminent logicians of our time, including Rami's advisor, Saharon Shelah, who proved a famous conjecture by Morley (1965). More on the Spectrum Problem can be found on the associated \href{https://en.wikipedia.org/wiki/Spectrum_of_a_theory#:~:text=More%20precisely%2C%20for%20any%20complete,of%20a%20countable%20theory%20T.}{Wikipedia page}, and while this is not the most authoritative source, its contents are nonetheless interesting.

Morley also proved a famous conjecture by Łos from the 1950s, which since became known as Morley's Categoricity Theorem.

\begin{boxtheorem}[Morley's Categoricity Theorem, Morley 1965]
    Let $T$ be a theory in a language $\Lang$. Assume that $\abs{\Lang} \leq \aleph_{0}$. If $\exists \lambda > \aleph_{0}$ such that $T$ is $\lambda$-categorical, them $\forall \lambda > \aleph_{0}$, $T$ is $\lambda$-categorical.
\end{boxtheorem}

One of our objectives in this course is to prove Morley's Categoricity Theorem.

As a side note, Morley was initially a PhD student of Saunders MacLane's at the University of Chicago. Morley didn't initially finish his PhD, to the point of losing his stipend at Chicago, but somehow landed a job at Berkeley, where he proved this famous theorem. MacLane, a staunch category theorist, didn't believe Morley's work was quite enough to merit a PhD; nevertheless, after being persuaded by the then-nascent (and very excited) model theory community, he eventually relented and awarded Morley his degree.

Here, we end our discussion on the spectrum problem. Before proceeding further, we recall the basics of cardinal arithmetic.

\subsection{Cardinal Arithmetic}

We begin by introducing notation.

\begin{boxnotation}
    We denote by
    \begin{itemize}
        \item $\ZF$ the \ZFA\ of Set Theory
        \item $\AC$ the \ACA
        \item $\ZFC$ the \ZFCA
    \end{itemize}
\end{boxnotation}

We denote cardinality of a set $A$ by $\abs{A}$ or $\card{A}$ and write $\abs{A} = \abs{B}$ if and only if there is a bijection from $A$ to $B$. Informally, a \textbf{cardinal} is a measure of cardinality. That is, a set $\lambda$ is a cardinal if $\lambda = \abs{A}$ for some set $A$. We denote by $\aleph_{0}$ the cardinal of the natural numbers, which we will denote $\omega$ in any cardinal- or ordinal-theoretic context.

There are more precise ways in which we can define the notions of ordinals and cardinals. We do not do this here, but we mention that there is an appendix in Rami's book and several sections in my undergrad logic lecture notes\todo{Add references} that discuss this.

\begin{boxdefinition}[Cardinal Arithmetic]
    Let $\lambda, \mu$ be cardinals, with $\lambda = \abs{A}$ and $\mu = \abs{B}$. We denote
    \begin{align*}
        \lambda + \mu &:= \sorry \\
        \lambda \cdot \mu &:= \abs{A \times B}
    \end{align*}
\end{boxdefinition}

The following is a famed theorem of Tarski, a direct consequence of which is precisely the fundamental theorem of cardinal arithmetic.

\begin{boxtheorem}[Tarski]
    We can make the following deduction:
    \begin{align*}
        \ZF \vdash \parenth{\AC \lr \forall A, \, \abs{A} \geq \aleph_{0} \to \abs{A \times A} = \abs{A}}
    \end{align*}
    Equivalently,
    \begin{align*}
        \ZF \vdash \parenth{\AC \lr \forall A, \, \lambda \geq \aleph_{0} \to \lambda \cdot \lambda = \lambda}
    \end{align*}
\end{boxtheorem}

The fundamental theorem of cardinal arithmetic, which states that $\abs{\omega \times \omega} = \abs{\omega}$, is clearly just the specialisation of the above result to the case where $\lambda = \aleph_{0}$.

There is another fact that will be important for our purposes.

\begin{boxtheorem}
    For infinite cardinals $\lambda, \mu \geq \aleph_{0}$, we have
    \begin{align*}
        \lambda \cdot \mu = \max\of{\lambda, \mu} = \lambda + \mu
    \end{align*}
\end{boxtheorem}

The reason for discussing cardinal arithmetic is that we can exploit it to prove the existence of submodels of specific cardinalities.

\section{A Word on Submodels}

Fix a language $\Lang$.

\subsection{Submodel Existence}

We begin by defining the cardinality of a structure.

\begin{boxdefinition}[Cardinality of a Structure]
    Let $N$ be a $\Lang$-structure. We define $\card{N}$ to be the cardinality of the union of 
\end{boxdefinition}

We begin with the famed submodel theorem.

\begin{boxtheorem}[The Submodel Theorem {\cite[Theorem 2.8, pp. 49-50]{MOAB}}]
    Let $M$ be a $\Lang$-structure. Define $\lambda := \abs{\Lang} + \aleph_{0}$. If $A \leq \abs{M}$, then there exists a substructure $N \leq M$ such that
    \begin{enumerate}[label = (\alph*)]
        \item $\card{N} \geq A$
        \item $\card{\abs{N}} \leq \abs{A} + \lambda$
    \end{enumerate}
\end{boxtheorem}
\begin{proof}
    By recursion on $n < \omega$, define sets $\setst{B_{n} \ssq \abs{M}}{n < \omega}$ such that
    \begin{enumerate}
        \item $B_{0} = \setst{c \in \Consts^M}{c \text{ is a constant symbol of } c} \cup A$
        \item If $n < \omega$, then $\abs{B_n} \leq \abs{A} + \lambda$
        \item For all $n < \omega$, define $B_{n+1} := \setst{F^M\of{\overline{a}}}{\overline{a} \in B_{n}} \cup B_n$
    \end{enumerate}
    This is enough: if we have such a sequence of $B_n$, then we could take $B := \bigcup_{n < \omega} B_n$ and define $N := \cycl{B, F^M, R^M, C^M}$. We can show that this satisfies the desired conditions.
    \begin{enumerate}[label = (\alph*)]
        \item \sorry
        \item \sorry
    \end{enumerate}

    Given these, all that remains now is to show that this is possible. \sorry\todo{Finish using textbook proof}
\end{proof}

\subsection{Elementary Submodels}

Recall the definition of elementary substructures (\sorry). In this subsection, we define an analogous notion for models.

\begin{boxdefinition}[Elementary Submodels]
    Let $M, N$ be $\Lang$-structures. We say that \textbf{$M$ is an elementary submodel of $N$}, denoted $M \preceq N$, if
    \begin{enumerate}
        \item $M \subseteq N$
        \item $M \models \varphi\!\brac{a_1, \ldots, a_n}$ iff $N \models \varphi\!\brac{a_1, \ldots, a_n}$ for every $\varphi \in \Fml\of{\Lang}$ and $a_1, \ldots, a_n \in \abs{M}$.
    \end{enumerate}
\end{boxdefinition}

We can relate this to the notion of elementary substructures in the following manner.

\begin{boxtheorem}[Tarski-Vaught 1956]
    Let $M, N$ be $\Lang$-structures with $M \subseteq N$. If $M \preceq N$, then $M \equiv N$.
\end{boxtheorem}

The converse is not true.

\begin{boxcexample}
    \sorry
    % If $M \equiv N$, then it is not necessary that $M \preceq N$.
\end{boxcexample}

\subsection{Chains of Substructures}

Throughout this subsection, fix a linear order $\parenth{\I, \le}$.

\begin{boxdefinition}[Chain]
    Let $\setst{M_{i}}{i \in \I}$ be $\Lang$-structures. We say they form a \textbf{chain} if for all $i_1, i_2 \in \I$, if $i_1 < i_2$ then $M_{i_1} \subseteq M_{i_2}$.
\end{boxdefinition}

For the remainder of this subsection, fix a chain $\setst{M_{i}}{i \in \I}$. It is easy to show that
\begin{align*}
    N := \bigcup_{i \in \I} M_i
\end{align*}
is an $\Lang$-structure as well. Moreover, $M_i \subseteq N$ for all $i \in \I$.

We can ask ourselves a natural question: suppose $T$ is a theory in $\Lang$ and that $\forall i \in \I$, $M_i \models T$. Is it necessarily true that $N \models T$ as well?

The answer turns out to be no when $\abs{I} \geq \aleph_0$.

\begin{boxcexample}
    Take $I = \omega$ and 
\end{boxcexample}

\subsection{The Löwenheim-Skolem Theorems}