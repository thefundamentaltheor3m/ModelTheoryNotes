\section{Models of Peano Arithmetic}

In this section, we build many models of Peano arithmetic, the most standard and universal of which is $\omega$, the natural numbers.

\subsection{The Language and Theory of Peano Arithmetic}

In this subsection, we set up our study of Peano arithmetic by defining the language in which we will work and the theory we will seek to model. We begin with some notation.

\begin{boxnotation}
    Let $\PA$ denote the theory of Peano arithmetic expressed in a language $L$, the `language of Peano arithmetic'.
\end{boxnotation}

\sorry

\subsection{Existence of Non-Standard Models of Peano Arithmetic}

We know that there is a standard model of Peano arithmetic, denoted $N$. This consists intuitively of the natural numbers---that is, ordinals less than $\omega$---with the usual addition, multiplication, additive and multiplicative identities. It turns out there are also other models, whose existence we will see in this subsection.

Before that, we mention that any model of Peano arithmetic admits a linear order.

\begin{boxtheorem}
    The following is a valid deduction.
    \begin{align*}
        \PA \vdash \forall x \forall y \exists z \parenth{y = x + z}
    \end{align*}
\end{boxtheorem}

\begin{boxtheorem}
    There exists some $M\models \PA$ such that $M$ is not isomorphic to the standard model $N$.
\end{boxtheorem}
\begin{proof}
    Augment the language $L$ to the $L_1 = L_{\PA} \cap \set{c}$ by adding a new constant symbol $c$. Let
    \begin{align*}
        T_1 := \PA \cup \set{c \neq 0} \cup \setst{c \neq \underbrace{1 + \cdots + 1}_{n \text{ times}}}{n < \omega}
    \end{align*}
    We show, using the Compactness Theorem (\sorry), that there is a model $M$ of $T_1$. Given $T_0 \subseteq T_1$ finite, let $n_0 < \omega$ be the largest natural number such that
    \begin{align*}
        c \neq \underbrace{1 + \cdots + 1}_{n \text{ times}}
    \end{align*}
    lies in $T_0$. Define the $L_1$-structure
    \begin{align*}
        M_0 := \cycl{\omega, +, \cdot, 0, 1, a}
    \end{align*}
    be the expansion of $N$ to $L_1$., with $c^{M_0} = a$. Then, $M_0 \models T_0$. Hence, since every finite subset of $T_1$ has a model, so does $T_1$. Call this model $M_1$. Denote by $M$ the restriction of $M_1$ to $L_{\PA}$.

    Now that we have established the existence of $M$, all that remains is to show that $M$ is not isomorphic to $N$. Let $b = c^M$. Suppose $M \cong N$ via an isomorphism $f$. Then, $f(b) \in \omega$, so there is some $n < \omega$ such that
    \begin{align*}
        f(b) = \underbrace{1 + \cdots + 1}_{n \text{ times}}
    \end{align*}
   
\end{proof} % I'm slightly unsure about interpreting c in M as b, because c is a symbol in the larger language L_1, not L... I get what you're going for (makes a lot of sense) but I'm stuck on the previous step - defining b. Will maybe ask Rami after class

%asking would be good - as I understand, all we need for the prooof is that there is an interpretation of c in M (c arises as a constant in the larger language, but nonetheless the existence of an interpretation implies that such an element exists in M) - I think now he's working on something else

% Makes sense. Yeah this might be something else

%we can check after

% (Unless you know?)
%(lmao)
% Oh wait. His proof isn't over

\subsection{Famous Results on Peano Arithmetic}

Recall the definition of the \textit{spectrum of a theory} (\Cref{Ch1:Def:Spectrum}). It turns out, we can use the idea of a spectrum to say something rather sophisticated about countable models of Peano Arithmetic.

\begin{boxtheorem}\label{Ch1:Thm:Spectrum_of_PA}
    $I\of{\aleph_0, \PA} = 2^{\aleph_0}$.
\end{boxtheorem}

We know that $\PA$ admits a model (indeed, a standard model). Therefore, we know that $\PA$ is a consistent theory. However, Gödel famously showed that it is not possible to prove consistency of $\PA$ using $\PA$ alone. That is, he proved the following.

\begin{boxtheorem}[Gödel's Second Incompleteness Theorem]
    $\PA \not\vdash \PA \text{ is consistent}$
\end{boxtheorem}
\begin{remark}
    Note that no number theorist would accept this as an example of a statement of interest in number theory which is not provable in $\PA$ (so says the model theorist). 
\end{remark}

Another interesting impossibility result involves Ramsey theorem.

\begin{boxtheorem}[Paris-Harrington 1976]
    $\PA \not\vdash \text{A special case of Ramsey's Theorem}$.
\end{boxtheorem}

\subsection{True Arithmetic and the Twin Prime Conjecture}

Recall the definition of the theory of a model (\Cref{Ch1:Def:Thy_of_Model}). Since $\cycl{\omega, +, \cdot, 0, 1}$ is obviously a model of Peano Arithmetic, its \textit{theory} strictly contains the theory of Peano Arithmetic. We call this new theory the theory of true arithmetic.

\begin{boxdefinition}[True Arithmetic]\label{Ch1:Def:True_Arith}
    Define the \textbf{theory of true arithmetic}, denoted $\TA$, to be
    \begin{align*}
        \TA := \Th\of{\cycl{\omega, +, \cdot, 0, 1}} \supsetneq \PA
    \end{align*}
\end{boxdefinition}

The following is known about the spectrum of true arithmetic.

\begin{boxtheorem}\label{Ch1:Thm:Spectrum_of_TA}
    For all cardinals $\lambda \geq \aleph_0$, $I\of{\lambda, \TA} = 2^{\lambda}$.
\end{boxtheorem}

Indeed, taking $\lambda = \abs{\omega} = \aleph_0$ gives us something reminiscent of \Cref{Ch1:Thm:Spectrum_of_PA}.

For the remainder of this subsection, we will talk about how we can use the theory of true arithmetic to study the twin prime conjecture. We begin with notation.

\begin{boxlnotation}
    Denote by $\psi$ the formula in the language of true arithmetic expressing that there are infinitely many twin primes. Let $\calP$ denote the set of prime numbers, a subset of $\omega$. We will also use the symbol $\mid$ in an in-fix manner to denote
    \begin{align*}
        x \mid y \iff \exists x \brac{y = z \cdot x}
    \end{align*}
\end{boxlnotation}

\begin{boxlemma}
    For every $S \ssq \calP$, there is a countable model $M_S \models \TA$. Moreover, $\exists a_S \in \abs{M_S}$ such that for all $p \in \calP$, $M_S \models p \mid a_S$ if and only if $p \in S$.
\end{boxlemma}
\begin{proof}
    Let $c$
\end{proof}