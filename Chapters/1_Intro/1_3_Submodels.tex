\section{A Word on Submodels}

Fix a language $\Lang$.

\subsection{Submodel Existence}

We begin by defining the cardinality of a structure.

\begin{boxdefinition}[Cardinality of a Structure]
    Let $N$ be a $\Lang$-structure. We define $\card{N}$ to be the cardinality of the union of $\textbf{F}^N(\Lang)\cup \textbf{C}^N(\Lang)\cup \textbf{R}^N(\Lang)\cup |N|$.
\end{boxdefinition}

We begin with the famed submodel theorem.

\begin{boxtheorem}[The Submodel Theorem {\cite[Theorem 2.8, pp. 49-50]{MOAB}}]\label{Ch1:Thm:Submodel_Thm}
    Let $M$ be a $\Lang$-structure. Define $\lambda := \abs{\Lang} + \aleph_{0}$. If $A \leq \abs{M}$, then there exists a substructure $N \leq M$ such that
    \begin{enumerate}[label = (\alph*)]
        \item $\card{N} \geq A$
        \item $\card{\abs{N}} \leq \abs{A} + \lambda$
    \end{enumerate}
\end{boxtheorem}
\begin{proof}
    By recursion on $n < \omega$, define sets $\setst{B_{n} \ssq \abs{M}}{n < \omega}$ such that
    \begin{enumerate}
        \item $B_{0} = \setst{c \in \Consts^M}{c \text{ is a constant symbol of } c} \cup A$
        \item If $n < \omega$, then $\abs{B_n} \leq \abs{A} + \lambda$
        \item For all $n < \omega$, define $B_{n+1} := \setst{F^M\of{\overline{a}}}{\overline{a} \in B_{n}} \cup B_n$
    \end{enumerate}
    This is enough: if we have such a sequence of $B_n$, then we could take $B := \bigcup_{n < \omega} B_n$ and define $N := \cycl{B, F^M, R^M, C^M}$. We can show that this satisfies the desired conditions.
    \begin{enumerate}[label = (\alph*)]
        \item \sorry
        \item \sorry
    \end{enumerate}

    Given these, all that remains now is to show that this is possible. \sorry\todo{Finish using textbook proof}
\end{proof}

\subsection{Elementary Submodels}

Recall the definition of elementary substructures (\sorry). In this subsection, we define an analogous notion for models.

\begin{boxdefinition}[Elementary Submodels]
    Let $M, N$ be $\Lang$-structures. We say that \textbf{$M$ is an elementary submodel of $N$}, denoted $M \preceq N$, if
    \begin{enumerate}
        \item $M \subseteq N$
        \item $M \models \varphi\!\brac{a_1, \ldots, a_n}$ iff $N \models \varphi\!\brac{a_1, \ldots, a_n}$ for every $\varphi \in \Fml\of{\Lang}$ and $a_1, \ldots, a_n \in \abs{M}$.
    \end{enumerate}
\end{boxdefinition}

We can relate this to the notion of elementary substructures in the following manner.

\begin{boxtheorem}[Tarski-Vaught 1956]
    Let $M, N$ be $\Lang$-structures with $M \subseteq N$. If $M \preceq N$, then $M \equiv N$.
\end{boxtheorem}

The converse is not true.

\begin{boxcexample}
    \sorry
    % If $M \equiv N$, then it is not necessary that $M \preceq N$.
\end{boxcexample}

\subsection{The Tarski-Vaught Test}

In this subsection, we explore a monumental result by Tarski and Vaught that gives a sufficient and necessary condition for a substructure to be elementary.

We begin by introducing some notation.

\begin{boxlnotation}
    Denote by $\star_{\psi}$ the statement
    \begin{align}
        \text{For all } a_1, \ldots, a_n \in \abs{M},
        \qquad
        M \models \psi\!\brac{a_1, \ldots, a_n}
        \iff
        N \models \psi\!\brac{a_1, \ldots, a_n}
    \end{align}
    for some $\psi \in \Fml(L)$.
\end{boxlnotation}

Next, we note a fact about substructures.

\begin{boxlemma}\label{Ch1:Lemma:Tarski-Vaught-star-psi}
    Let $N$ be an $L$-structure and let $M \ssq N$ be a substructure of $N$. Then, $\star_{\psi}$ holds for all quantifier-free formulae $\psi \in \Fml(L)$.
\end{boxlemma}
\begin{proof}
    \sorry
\end{proof}

\begin{boxtheorem}[The Tarski-Vaught Test]\label{Ch1:Thm:Tarski-Vaught}
    Let $M, N$ be $L$-structures with $M \ssq N$. Then, the following are equivalent.
    \begin{enumerate}
        \item $M \preceq N$
        \item If, for every $\varphi\of{y, x_1, \ldots, x_n} \in \Fml(L)$ and $a_1, \ldots, a_n \in \abs{M}$,
        \begin{align*}
            N \models \exists y \, \varphi\of{y ,a_1, \ldots, a_n}
        \end{align*}
        then there is some $b \in \abs{M}$ such that $N \models \varphi\!\brac{b, a_1, \ldots, a_n}$
    \end{enumerate}
\end{boxtheorem}
\begin{remark}
    We can see this as 'a more ``algebraic'' notion of being a submodel.' What is a formula? A list of quantifiers, connectives, etc. - for example, we can think of polynomials in several variables, which we wish to solve. If $\vp(y,x)$ is a set of finitely many equations (which we wish to solve), we can see this result as telling us that if there exists a solution to the system $y\in N$, there is also a $b$ in the substructure $M$ which also solves the same system. 
\end{remark}
\begin{proof}[Proof of \Cref{Ch1:Thm:Tarski-Vaught}.]
    \begin{description}
        \item[\underline{$1 \implies 2$.}]
        Fix  $\varphi\of{y, x_1, \ldots, x_n} \in \Fml(L)$ and $a_1, \ldots, a_n \in \abs{M}$. Suppose
        \begin{align*}
            N \models \exists y \, \varphi\of{y ,a_1, \ldots, a_n}
        \end{align*}
        Since $M \preceq N$, by definition of satisfaction, we know that
        \begin{align*}
            M \models \exists y \, \varphi\of{y ,a_1, \ldots, a_n}
        \end{align*}
        This tells us that there is $b \in \abs{M}$ witnessing $\varphi$, meaning that
        \begin{align*}
            M \models \exists y \, \varphi\of{b ,a_1, \ldots, a_n}
        \end{align*}
        Then, since $M \preceq N$, we have that
        \begin{align*}
            N \models \exists y \, \varphi\of{b ,a_1, \ldots, a_n}
        \end{align*}
        as required.

        \item[\underline{$2 \implies 1$.}]
        We show, by induction on $\varphi\of{y, x_1, \ldots, x_n} \in \Fml(L)$, that $\star_{\psi}$ holds for all $\psi \in \Fml(L)$. Recall, from \Cref{Ch1:Lemma:Tarski-Vaught-star-psi}, that $\star_{\psi}$ does hold for quantifier-free formulae $\psi$. In particular, it holds for atomic formulae. We can now consider the different possible cases on $\psi$.
        \begin{enumerate}
            \item \underline{$\psi$ is of the form $\psi_1 \land \psi_2$.} \newline
            \sorry

            \item \underline{$\psi\of{x_1, \ldots, x_n}$ is of the form $\exists y \, \varphi\of{y, x_1, \ldots, x_n}$.} \newline
            Assume that $M \models \psi\!\brac{a_1, \ldots, a_n}$. Then, by assumption, there is some $b \in \abs{M}$ such that $M \models \varphi\!\brac{b, a_1, \ldots, a_n}$. Then, $N \models \varphi\!\brac{b, a_1, \ldots, a_n}$ because $b \in \abs{M} \ssq \abs{N}$. Thus,
            \begin{align*}
                N \models \exists y \, \varphi\of{y, a_1, \ldots, a_n}
            \end{align*}
            ie,
            \begin{align*}
                N \models \psi\!\brac{a_1, \ldots, a_n}
            \end{align*}
        \end{enumerate}
    \end{description}
    By $\star_{\psi}$, $b \in \abs{M}$ % The truth is I'm a bit confused and a little lost - need to think a little about the reasoning here. Should be ok but need to think about it a bit
\end{proof} %...not alone (heh :)) - there's probably time (which will be made) to do just that :D Sounds good

\subsection{Chains of Substructures}

Throughout this subsection, fix a linear order $\parenth{\I, \le}$.

\begin{boxdefinition}[Chain]
    Let $\setst{M_{i}}{i \in \I}$ be $\Lang$-structures. We say they form a \textbf{chain} if for all $i_1, i_2 \in \I$, if $i_1 < i_2$ then $M_{i_1} \subseteq M_{i_2}$.
\end{boxdefinition}

\begin{boxlnotation}
    For the remainder of this subsection, fix a chain $\setst{M_{i}}{i \in \I}$. Define
    \begin{align*}
        N := \bigcup_{i \in \I} M_i
    \end{align*}
\end{boxlnotation}
It is easy to show that $N$ is an $\Lang$-structure as well. Moreover, $M_i \subseteq N$ for all $i \in \I$.

We can ask ourselves a natural question: suppose $T$ is a theory in $\Lang$ and that $\forall i \in \I$, $M_i \models T$. Is it necessarily true that $N \models T$ as well?

The answer turns out to be no when $\abs{I} \geq \aleph_0$.

\begin{boxcexample}
    Take $I = \omega$ and \sorry
\end{boxcexample}

We can instead define elementary chains, which are the analogues of chains for elementary substructures.

%just not the elementary part? yeah you're right (yes, we do have chains)
\begin{boxdefinition}[Elementary Chain] % I got this dw
    We say that $\setst{M_{i}}{i \in \I}$ form an \textbf{elementary chain} if for all $i_1, i_2 \in \I$, if $i_1 < i_2$ then $M_{i_1} \preceq M_{i_2}$.
\end{boxdefinition}

We can apply \Cref{Ch1:Thm:Tarski-Vaught} to prove an important result on elementary chains.

\begin{boxtheorem}[Tarski-Vaught Chain Theorem]
    Assume $\setst{M_i}{i \in I}$ is an elementary chain. Let $N$ be an $L$-structure. Then, the $L$-structure
    \begin{align*}
        N = \bigcup_{i \in I} M_i
    \end{align*}
    satisfies the property that for all $i \in I$, $M_i \preceq N$.
\end{boxtheorem}
\begin{proof}
    Since we already know that $M_i \ssq N$, it is enough to show that for every $\psi\of{x_1, \ldots, x_n} \in \Fml(L)$ and every $i \in I$, $\star_{\psi}$ holds (where $\star_{\psi}$ is the formula defined as local notation in the previous subsection, considered along with the substructure $M_i$ of $N$).

    We know that for all $i \in I$, since $M_i \ssq N$, $\star_{\psi}$ holds for atomic formula. We induct on logical connectives and quantifiers to exhaustively prove that the statement $\forall i \in I, \star_{\psi}$ is true. We only do a few cases explicitly.

    \begin{enumerate}
        \item \underline{$\psi\of{x_1, \ldots, x_n}$ is of the form $\neg \vp\of{x_1, \ldots, x_n}$.}

        Then, for all $i \in I$, $M_i \models \psi\!\brac{a_1, \ldots, a_n}$ iff $M_i \not\models \vp\!\brac{a_1, \ldots, a_n}$. By the induction hypothesis that $\forall i \in I, \star_{\phi}$ holds for all formulae $\phi$ with fewer quantifiers than $\psi$, we can conclude that
        \begin{align*}
            M_i \not\models \vp\!\brac{a_1, \ldots, a_n}
            \iff
            N \not\models \vp\!\brac{a_1, \ldots, a_n}
        \end{align*}
        This tells us that $N \models \psi\!\brac{a_1, \ldots, a_n}$ for every interpretation $a_i$ of $x_i$.

        \item \underline{$\psi\of{x_1, \ldots, x_n}$ is of the form $\exists y\,  \vp\of{y, x_1, \ldots, x_n}$.}

        Fix $i \in I$. Then, $M_i \models \psi\!\brac{a_1, \ldots, a_n} \implies M_i \models \exists y \, \varphi\!\of{y, a_1, \ldots, a_n}$. Then, there is some $b \in \abs{M_i}$ such that $M_i \models \varphi\!\brac{b, a_1, \ldots, a_n}$. This tells us, by the induction hypothesis, that $N \models \varphi\!\brac{b, a_1, \ldots, a_n}$ for some $b \in \abs{M_i} \ssq \abs{N}$. Thus, $N \models \exists y\,  \vp\of{y, x_1, \ldots, x_n}$, meaning $N \models \psi\!\brac{a_1, \ldots, a_n}$ as required.
    \end{enumerate}
    We can argue similarly for other quantifiers and connectives.
\end{proof}

\begin{boxcorollary}
    If $T$ is an $L$-theory, then if $M_i \models T$ for all $i \in I$ then $N \models T$ as well.
\end{boxcorollary}

We don't prove this corollary here.

We finally mention an additional nuance. Suppose $i \in I$ satisfies $a_1, \ldots, a_n \in \abs{M_i}$ and $N \models \psi\!\brac{a_1, \ldots, a_n}$. Say that $\psi$ is of the form $\exists y \, \varphi\of{y, x_1, \ldots, x_n}$. Then, by the definition of satisfaction, we know that there is some $b \in \abs{N}$

\begin{boxdefinition}[directed poset]
    Let $(I, <)$ be a poset, we say that $I$ is ``directed" if $\forall i, j\in I$ there exists $b\in I$ such that $i\leq k, j\leq k$.
\end{boxdefinition}
\begin{remark}
    This theorem can be extended - we don't need a linearly ordered set, possibly a directed poset would suffice? (We won't see this in this course.)
\end{remark}

%this is why elementary submodel is stronger than having the same theory(?) (just getting all the terms straight in my head, sorry again) - alright, no worries (I will continue doing that)
% Hmm - I might have them mixed up asw, need to look at it again. (I want to say yes) - Lmk once you figure it out!
% Other way round - sentences are formulae without free vars - sure

\subsection{Restrictions and Expansions} % Move to section on submodels!!!!

Before going any further, we will need to define a central tool: restrictions and expansions. Throughout this subsection, fix a language $L$ and an $L$-structure $M$.

\begin{boxdefinition}[Restriction/Expansion]\label{Ch1:Def:Res_Exp}
    Let $L_1 \subseteq L$, so that $L$ contains relations, functions, and constants $\Rels\of{L_1}$, $\Funcs\of{L_1}$, and $\Consts\of{L_1}$. The \textbf{restriction of $M$ to $L$} is the $L_1$-structure
    \begin{align*}
        M \vert_{L_1} := \cycl{\abs{M}, \Rels^{M}\of{L_1}, \Funcs^M\of{L_1}, \Consts^M\of{L_1}}
    \end{align*}
    Dually, we say that \textbf{$M$ is the expansion of $M\vert_{L_1}$ to $L$}.
\end{boxdefinition}

A good example of this is to model-theoretically encode the fact that every field is also an abelian group (additively).

\begin{boxexample}[Restriction: Fields to Abelian Groups]
    Let $L$ be the language of fields and let $L_1$ be the language of (additively expressed) abelian groups. Then, if
    \begin{align*}
        M = \cycl{\Q, +, \cdot, 0, 1}
    \end{align*}
    then its restriction to $L_1$ is
    \begin{align*}
        M\vert_{L_1} = \cycl{\Q, +, 0}
    \end{align*}
\end{boxexample}

\subsection{The Löwenheim-Skolem Theorems}

Throughout, let $L$ be a language.

We begin with the downwards theorem, which gives us substructures with control over cardinality. It looks similar to the Submodel Theorem (\Cref{Ch1:Thm:Submodel_Thm}).

\begin{boxtheorem}[Downwards Löwenheim-Skolem-Tarski Theorem]\label{Ch1:Thm:Downwards_LS}
    Let $M$ be an $L$-structure. Define $\lambda := \abs{L} + \aleph_0$. For all $A \ssq \abs{M}$, there is some $N \preceq M$ with $\abs{N} \supseteq A$ and $\abs{\abs{N}} \leq \abs{A} + \lambda$, where $\abs{\abs{N}}$ refers to the cardinality of the universe of $N$.
\end{boxtheorem}
Recall that \Cref{Ch1:Thm:Submodel_Thm} gives us the existence of $N \leq M$ with the desired properties. The difference is that in \Cref{Ch1:Thm:Downwards_LS}, we have elementarity.
\begin{proof}[Proof of \Cref{Ch1:Thm:Downwards_LS}.]
    Fix $A \ssq \abs{M}$. Fix $\varphi\of{y, x_1, \ldots, x_n} \in \Fml(L)$. Fix a well-ordering $\leq$ of $\abs{M}$, which we pick using the Axiom of Choice.
    
    We define the function $G_{\varphi} : \abs{M} \times \cdots \times \abs{M} \to \abs{M}$ as follows: for all $b_1, \ldots, b_n$, define
    \begin{align}
        G_{\varphi}\of{b_1, \ldots, b_n}
        =
        \begin{cases}
            \min_{\leq}\!\abs{M} & \text{ if } M \not\models \exists y \, \varphi\of{y, x_1, \ldots, x_n} \\
            \min_{\leq}\!\setst{a \in \abs{M}}{M \models \varphi\!\brac{a, b_1, \ldots, b_n}} & \text{ if } M \models \exists y \, \varphi\of{y, x_1, \ldots, x_n}
        \end{cases}
        \label{Ch1:Eq:DLS_Skolem_Fn}
    \end{align}
    Thus, for every $b_1, \ldots, b_n \in \abs{M}$, $G_{\varphi}$ gives the least element $a \in \abs{M}$ such that $\varphi\!\brac{a, b_1, \ldots, b_n}$ is satisfied (and returns a junk value of there is no such element).

    We now augment our language $L$ in the following manner. Define
    \begin{align}
        L_1 := L \cup \setst{G_{\varphi}\of{x_1, \ldots, x_n}}{\varphi\of{y, x_1, \ldots, x_n} \in \Fml(L)}
        \label{Ch1:Eq:DLS_Lang_Augmentation}
    \end{align}
    That is, we create $L_1$ by adding to $L$ all the constant symbols that correspond to \sorry.

    Let $M_1$ be the expansion of $M$ to $L$ (cf. \Cref{Ch1:Def:Res_Exp}). Observe that
    \begin{align*}
        \abs{L_1} \leq \abs{L} + \abs{\Fml(L)} \leq \lambda + \aleph_0 \cdot \lambda = \lambda
    \end{align*}
    Now, we can apply the Submodel Theorem (\Cref{Ch1:Thm:Submodel_Thm}) to $M_1$ and $A$ to obtain some $N_1 \ssq M_1$ (as $L_1$-structures) such that $\abs{N_1} \supseteq A$ and $\abs{\abs{N}} \leq \abs{A} + \lambda$.

    Define $N := N_1\vert_{L}$, the restriction of $N_1$ to $L$. To show that $N$ has the properties desire, we really only need to show that $N\vert_L \preceq M \vert_L$. We do this by using the Tarski-Vaught test (\Cref{Ch1:Thm:Tarski-Vaught}) to prove that $M_1\vert_{L} \preceq N_1\vert_{L}$.

    Fix $\psi\of{y, x_1, \ldots, x_n} \in \Fml(L)$. Suppose that $M \models \exists y \, \psi\of{y, a_1, \ldots, a_n}$ for some $a_1, \ldots, a_n \in \abs{N_1}$. Then, by definition of $G$ for formulae, we have that $G_{\psi}\of{a_1, \ldots, a_n} \in \abs{M_1}$ is the smallest $b \in \abs{M_1}$ such that $M_1 \models \psi\!\brac{b, a_1, \ldots, a_n}$. Since $a_1, \ldots, a_n \in \abs{N_1}$, and since $N$ is closed under taking $G_{\psi}$, we must have $b \in N_1$.
    
    % This is so entertaining fr
    %very - also good to know that's one way to get a PhD - we'll see (also sorry, thank you for scribing)
    % No problem - This is a not unusual but weird-looking (to me) argument, seems somewhat... not artificial, but... not quite sure what word to use...
    % Haha
\end{proof}

We will find that the techniques of defining functions like $G_{\varphi}$ in \eqref{Ch1:Eq:DLS_Skolem_Fn} and augmenting languages as in \eqref{Ch1:Eq:DLS_Lang_Augmentation} will come up time and time again in model theory.

\begin{boxtheorem}[Upwards Löwenheim-Skolem Theorem]
    Suppose $T$ has an infinite model. Then given any cardinal $\lambda \geq \aleph_0+|\Lang(T)|$ there exists some $L$-structure $M$ of cardinality $||M||=\lambda$ and $M\models T$.
\end{boxtheorem}

\begin{proof}
    We will just add $\lambda$ many constants to a model of $T$. In particular, we let $T^k:=T\cup\{c_i\neq c_j|i\neq j<\lambda\}$. By the compactness theorem $T^k$ has a model $N$, and if we now let $A=\{c_i^M| i<\lambda\}$ then applying Downward Lowenheim-Skolem Tarski we can obtain some $M<N\restriction \Lang(T)$ of cardinality $\lambda$ (completing the proof).
\end{proof}

\subsection{Complete and Elementary Diagrams}

\begin{boxdefinition}[Complete Diagram on $M$]
    Let $M$ be an $L$-structure and $L_M:=\Lang(M)\cup \{c_a|a\in M\}$. Letting $M':=\{ M, c_a\}_{a\in |M|}$ with the constants interpreted such that $c_a^{M'}=a$ for all $a\in |M|$, we define $\CD(M)=Th(M')$ denote the ``complete diagram of $M$.''
 \end{boxdefinition}

 \begin{boxdefinition}[Elementary Diagram of $M$] 
     We now define $$\ED(M)=\{\vp \in \CD(M)|\vp\text{ is quantifier free}\}$$
 \end{boxdefinition}

 \begin{boxlemma}[Lemma 1]
     Suppose $N\models \ED(M)$ and let $N^*:=N\restriction L$. Then there exists $\vp:|M|\rightarrow |N^*|$ an injective homomorphism (not necessarily injective).
 \end{boxlemma}
 For a proof, we can just take $\vp(a)=c_a^{N^*}$.

 \begin{boxlemma}[Lemma 2]
     Suppose $N\models \CD(M)$ with $N^*:= N\restriction L$. The there exists $\vp:|M|\rightarrow |N|$ an elementary embedding [i.e. $\vp[M]\prec N$].
 \end{boxlemma}

 As an application, given $M$ and some $\Gamma$ a set of sentences, if $\CD(M)\cup \Gamma$ which is consistent, it follows that there must exist some $N$ such that $M$ is an elementary submodel of $N$ and $N\models \Gamma$.

 \begin{boxtheorem}[Upward Lowenheim-Skolem-Tarski]
     Given an infinite $L$-structure $M$ and any cardinal $\lambda\geq ||M||+|\Lang(M)|$ there exists some $N>M$ of cardinality $\lambda$.
 \end{boxtheorem}
 For this one-line proof, just apply \sorry to $\CD(M)$.

 What are the basic theorems of model theory? ULS, DLST, and Compactness Theorem - and it turn out that these are equivalent to AC in ZFC!

\begin{remark}
     A fact proved by Lauchli and Levi is that $ZF\vdash [CT\leftrightarrow BPI]$ where $BPI$ deals with ``Boolean Prime Ideals'' \sorry (see book), and further in $ZF$ all of this is equivalent to Tychonoff's Theorem. The effect of this is that it's very difficult to do interesting model theory without choice.
\end{remark}
\begin{remark}
    ``I feel it in my bones, it is true.'' - Shelah's response to (Grossberg's) the question of how one can feel comfortable using Axiom of Choice in Model Theory. Made quite an impression on Professor Grossberg.
\end{remark}

