\section{A Crash Course on First-Order Logic}

We begin by recalling the notion of a first-order language.

\begin{boxdefinition}[Language]
    A \textbf{language} is a disjoint union
    \begin{align*}
        \Lang = \Funcs \cup \Rels \cup \Consts
    \end{align*}
    of countable sets of symbols, where
    \begin{itemize}
        \item $\Funcs$ is a set of function symbols
        \item $\Rels$ is a set of relation symbols
        \item $\Consts$ is a set of constant symbols
    \end{itemize}
\end{boxdefinition}

Next, we recall the notion of a structure in a language.

\begin{boxdefinition}[Structure]
    Let $\Lang$ be a structure. An \textbf{$\Lang$-structure} is a tuple
    \begin{align*}
        M = \cycl{U; F, R, C}
    \end{align*}
    consisting of a non-empty set $U$ and functions, relations, and constants that live in $\Funcs$, $\Rels$ and $\Consts$ respectively. $U$ known as the \textbf{universe} of $M$, and is denoted by $\abs{M}$. The function, relation and constant symbols of $M$ are denoted $F^M$, $R^M$ and $C^M$ respectively.
\end{boxdefinition}

Any function or relation in a structure has an \textbf{arity}, which is informally the number of arguments it takes. An important fact to note is that arities are not a feature of functions and relations themselves, but of their corresponding \textit{symbols}. In other words, \textbf{arity is a syntactic notion}. Semantically speaking, when we seek an interpretation of a function symbol of some arity $n$, we are forced to limit our search to the set of functions from $U^n$ to $U$.

We now describe the notion of structure-preserving bijections, known as isomorphisms.

\begin{boxdefinition}[Isomorphism]
    Let $\Lang$ be a language and let $M, N$ be $\Lang$-structures. We say that a function $g : \abs{M} \to \abs{N}$ is an \textbf{isomorphism} if
    \begin{enumerate}
        \item $g$ is a bijection
        \item ``$g$ commutes with functions''
        \item ``$g$ commutes with relations''
    \end{enumerate}
    where the double-quotes for the second and third point above refer to the fact that we implicitly require an equality of arities condition before we can talk about composing isomorphisms with multi-ary functions.
\end{boxdefinition}

We will now see an example.

\begin{boxexample}[A Definition]
    You literally just saw one...
\end{boxexample}