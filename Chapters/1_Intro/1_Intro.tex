\chapter{An Introduction to Model Theory} \label{Ch1:CH}
\thispagestyle{empty}

Model Theory is one of the two main branches of mathematical logic, alongside Set Theory. We can view Model Theory as a translation of algebra to the world of set theory, or as a \textit{generalisation} of algebra, with many applications to algebra. Indeed, it is possible to view model theory as a more specialised version of category theory, at least in its power to deal with generality. We can prove tremendously deep theorems about algebra, or other fields of maths, purely using model theory.

% We also add a few historical comments that might be interesting to the reader (they were interesting to me). In the 1960s, there was a collection of papers published at two back-to-back maths conferences that together defined a field called ``model theory''. These papers sought to generalise an earlier work by Steinitz from 1910, which, over the course of 200 pages, made substantial progress on field theory, building on the work of the legendary Évariste Galois. In the 70s, more papers on model theory began to appear, generalising the work of Lowenheim and Skolem from the 1910s as well as work by a group based in Warsaw led by Tarski from the late 1920s. After the break of war in 1939 and the murder of several of Tarski's colleagues at the hands of the Nazis, Tarski took refuge in the United States, where he continued his work in UC Berkeley. 

Before we can talk about model theory in any detail, we need to recall a few notions from the world of logic.

\section{A Crash Course on First-Order Logic}

\subsection{Languages and Structures}

We begin by recalling the notion of a first-order language.

\begin{boxdefinition}[Language]
    A \textbf{language} is a disjoint union
    \begin{align*}
        \Lang = \Funcs \cup \Rels \cup \Consts
    \end{align*}
    of countable sets of symbols, where
    \begin{itemize}
        \item $\Funcs$ is a set of function symbols
        \item $\Rels$ is a set of relation symbols
        \item $\Consts$ is a set of constant symbols
    \end{itemize}
\end{boxdefinition}

Next, we recall the notion of a structure in a language.

\begin{boxdefinition}[Structure]
    Let $\Lang$ be a structure. An \textbf{$\Lang$-structure} is a tuple
    \begin{align*}
        M = \cycl{U; F, R, C}
    \end{align*}
    consisting of a non-empty set $U$ and functions, relations, and constants that live in $\Funcs$, $\Rels$ and $\Consts$ respectively. $U$ known as the \textbf{universe} of $M$, and is denoted by $\abs{M}$. The function, relation and constant symbols of $M$ are denoted $F^M$, $R^M$ and $C^M$ respectively.
\end{boxdefinition}

Any function or relation in a structure has an \textbf{arity}, which is informally the number of arguments it takes. An important fact to note is that arities are not a feature of functions and relations themselves, but of their corresponding \textit{symbols}. In other words, \textbf{arity is a syntactic notion}. Semantically speaking, when we seek an interpretation of a function symbol of some arity $n$, we are forced to limit our search to the set of functions from $U^n$ to $U$.

We now describe the notion of structure-preserving bijections, known as isomorphisms.\todo{MAKE MORE PRECISE - FOLLOW DEF 2.9 IN BOOK}

\begin{boxdefinition}[Isomorphism]\label{Ch1:Def:Isomorphism}
    Let $\Lang$ be a language and let $M, N$ be $\Lang$-structures. We say that a function $g : \abs{M} \to \abs{N}$ is an \textbf{isomorphism} if
    \begin{enumerate}
        \item $g$ is a bijection
        \item ``$g$ commutes with functions''
        \item ``$g$ commutes with relations''
        \item ``$g$ agrees on constants''
    \end{enumerate}
    where the double-quotes for the second and third point above refer to the fact that we implicitly require an equality of arities condition before we can talk about composing isomorphisms with multi-ary functions.
\end{boxdefinition}

% Lecture 2

We are now ready to define submodels.

\begin{boxdefinition}[Submodel]
    Let $M, N$ be $\Lang$-structures. We say that $M$ is a submodel of $N$, denoted $M \subseteq N$, if
    \begin{enumerate}
        \item $\abs{M} \subseteq \abs{N}$.
        \item For all function symbols $F\of{x_1, \ldots, x_n}$, the interpretation in $M$ agrees with the interpretation in $N$.
        \item For all relation symbols $R\of{a_1, \ldots, a_n}$, the interpretation in $M$ agrees with the interpretation in $N$.
        \item For all constant symbols $C$, the interpretation in $M$ agrees with the interpretation in $N$.
    \end{enumerate}
\end{boxdefinition}

In particular, a submodel of a model is also a model. For instance, if $G$ is a group, ie, a model of the group axioms, then any submodel of $G$ is, in fact, a subgroup of $G$, and a group in its own right (in that it again models the group axioms).

We now talk about more syntactic elements of a language.

\subsection{Syntax: Terms, Formulae, Sentences and Theories}

\begin{boxdefinition}[Terms]\label{Ch1:Def:Term}
    Let $\Lang$ be a language. $\Term(\Lang)$ is defined to be the minimal set of finite sequences of symbols\footnote{Here, the set of variable symbols is countable, but we might, on occasion, need uncountably many variable symbols} from
    \begin{align*}
        \set{(, ), [, ]} \cup \Consts \cup \Funcs \cup \set{x_1, x_2, x_3, \ldots}
    \end{align*}
    satisfying the following rules:
    \begin{enumerate}
        \item Every constant symbol is a term.
        \item Every variable is a term.
        \item For all $n$-ary functions $f$ and $n$ terms $t_1, \ldots, t_n$, $f\of{t_1, \ldots, t_n}$ is a term.
        \item Every term arises in this way.
    \end{enumerate}
\end{boxdefinition}
\begin{remark}%? do you mind if I write in remarks? you can delete them later
    In other words, the elements of $\Term(\Lang)$ are exactly the constants, variables, and functions of such (constants and variables).
\end{remark}

Recall, from \Cref{Ch1:Def:Isomorphism}, that isomorphisms are, in particular, bijections that agree on constants. It is possible to show that they also agree on the interpretations of terms in models. We will show this later.

% For an ULTRA-FORMAL presentation of the material, look at D. Monk Mathematical Logic, Springer GTM. But Rami's feedback is: an undergraduate cannot follow it because it is too technical, and a graduate student will find that it has too much detail but not much actual content.

In similar fashion, we can define the formulae in a language.

\begin{boxdefinition}[Formulae]
    Let $\Lang$ be a language. $\Fml(\Lang)$ is defined to be the minimal set of finite sequences of symbols from
    \begin{align*}
        \Term\of{L} \cup \set{\land, \lor, \to, \neg, \ldots} \cup \set{=} \cup \set{\forall, \exists}
    \end{align*}
    satisfying the following rules:
    \begin{enumerate}
        \item For all $\tau_1, \tau_2 \in \Term\of{\Lang}$, $\tau_1 = \tau_2$ is a formula.
        \item For all $n$-ary relations $R$ and $n$ terms $t_1, \ldots, t_n$, $R\of{t_1, \ldots, t_n}$ is a formula.
        \item For all connectives $\star$ and formulae $\Phi$ and $\Psi$, $\Phi \star \Psi$ is a formula.
        \item For a quantifier $Q$, variable $x$ and formula $\varphi(x)$, $Qx \parenth{\varphi\of{x}}$ is a formula.
        \item Every formula arises in this way.
    \end{enumerate}
    Formulae consisting only of a single relation symbol (including formulae that only consist of an equality) are called \textbf{atomic formulae}. The atomic formulae of $\Lang$ are denoted $\AFml(\Lang)$.
\end{boxdefinition}

For a formula $\varphi$, denote by $\FV{\varphi}$ the set of free variables of $\varphi$. It is sometimes useful to distinguish those formulae in a language that contain no free variables.

\begin{boxdefinition}
    Define the set of \textbf{sentences} of a language $\Lang$ to be
    \begin{align*}
        \Sent\of{\Lang} := \setst{\varphi \in \Fml\of{\Lang}}{\FV{\varphi} = \emptyset}
    \end{align*}
\end{boxdefinition}

Essentially, a formula with no free variables is called a sentence. A theory is simply a set of sentences.

\begin{boxdefinition}[Theory]
    Let $\Lang$ be a language. An $\Lang$-theory is any subset $T \subseteq \Sent(\Lang)$.
\end{boxdefinition}

There are many well-known theories in mathematics. The most familiar examples come from algebra.

\begin{boxexample}[The Theory of Fields]\label{Ch1:Eg:Thy_of_Fields}
    The theory of fields is a first-order theory in the language of fields. This is a language with function symbols $+, \times, \inv$, relation symbol $=$, and constant symbols $0$ and $1$. It also has other first-order symbols, such as quantifiers, connectives and punctuation, but we ignore these (indeed, we will always take for granted that these exist). The theory of fields consists of the following sentences in this language:
    \begin{enumerate}
        \item $\forall x \forall y \forall z \brac{\parenth{x + y} + z = x + \parenth{y + z}}$
        \item $\forall x \forall y \brac{x + y = y + x}$
        \item $\forall x \exists y \brac{x + y = 0} \land \forall x \brac{x + 0 = x}$
        \item $\forall x \brac{x \neq 0 \to \exists y \brac{x y = 1}}$, where $\neq$ is the obvious shorthand
        \item $\forall x \brac{x \times 1 = x}$
        \item $\forall x \forall y \forall z \brac{x \times \parenth{y + z} = x \times y + x \times z}$
    \end{enumerate}
    Collectively, these sentences are known as the \textbf{theory of fields}.
\end{boxexample}

There are numerous well-known examples of fields. There is the question of how we can formally describe what it means for some structure, such as the rational numbers, to \textit{satisfy} the above sentences. To that end, we introduce the semantics of interpretation, namely, the notion of \textbf{satisfaction}.

\subsection{Semantics: Satisfaction}

We begin with the most important definition of this entire section.

\begin{boxdefinition}[Satisfaction - Sentences]
    Let $\Lang$ be a language and $M$ an $\Lang$-structure. For any $\varphi \in \Fml\of{\Lang}$, we say that \textbf{$M$ models $\varphi$}, denoted $M \models \varphi$, if \sorry % This is soooo tedious fr
\end{boxdefinition}

\begin{remark} % Move this to after the definition of satisfaction for sentences (once you finish). Thanks!!
     We note that, (a) $M \vDash \forall x \varphi(x)\iff N \vDash \neg \exists x \neg \varphi(x)$ and (b) $M \vDash \exists x \varphi(x) \iff M \vDash \neg \forall x \neg \varphi(x)$. This is worthy of note as for proofs, we will often need to induct on (the number of symbols in) formulas - just as we only need the logical connectives $\neg, \vee$ to get the rest, we only need to check satisfcation for a single quantifier and $\neg$.
\end{remark} % sure (thank you!)

We can define satisfaction for theories in the obvious way.

\begin{boxdefinition}[Satisfaction - Theories]
    Let $\Lang$ be a language and $M$ an $\Lang$-structure. Given an $\Lang$-theory $T$, we say that $M \models T$ if $M \models \psi$ for all $\psi \in T$.
\end{boxdefinition}

We are now ready to state a simple-sounding but rather non-trivial result.

\begin{boxlemma}
    Suppose $M$ and $N$ are both $\Lang$-structures. If $M \subseteq N$, then for all $\tau \in \Term\of{\Lang}$, $\tau^{M}\!\brac{a} = \tau^N\!\brac{a}$, where $a \in \abs{M} \times \cdots \times \abs{M}$ and $\tau^M\!\brac{a}$ and $\tau^N\!\brac{a}$ denote interpretations of $\tau$ in $M$ and $N$ with the variables all being interpreted as the components of $a$.
\end{boxlemma}

We do not prove this result. It is not difficult.

Next is a less trivial result.

\begin{boxlemma}
    If $M \subseteq N$, then $M \models \varphi$ if and only if $N \models \varphi$ for all quantifier-free formulae $\varphi$.
\end{boxlemma}


Going back to \Cref{Ch1:Eg:Thy_of_Fields}, we can now say the following.

\begin{boxexample}[Models of the Theory of Fields]
    Recall the theory of fields, seen in \Cref{Ch1:Eg:Thy_of_Fields}. It can be shown that the following structures in the language of fields satisfy the theory of fields:
    \begin{align*}
        \Q, \R, \C, \quotient{\Z}{7\Z}
    \end{align*}
    A model of the theory of fields is known simply as a \textbf{field}. We all know that this is an incredibly rich theory. It might have been even richer if Évariste Galois had had better control over his faculties...
\end{boxexample}

\subsection{Elementary Equivalence}

Recall the definition of an isomorphism of structures (\Cref{Ch1:Def:Isomorphism}). We can regard isomorphism as a \textit{syntactic} notion of equivalence of structures. In this subsection, we explore a \textit{semantic} notion of equivalence of models.

\begin{boxdefinition}[Elementary Equivalence]
    Let $\Lang$ be a langauge and let $M$ and $N$ be $\Lang$-structures. We say \textbf{$M$ is elementarily equivalent to $N$} if for every sentence $\varphi \in \Sent\of{\Lang}$, we have that $M$ satisfies $\varphi$ if and only if $N$ satisfies $\varphi$. We denote this
    \begin{align*}
        M \equiv N
    \end{align*}
\end{boxdefinition}

We can now relate isomorphisms to equivalence in the following manner.

\begin{boxtheorem}
    Let $\Lang$ be a language and let $M$ and $N$ be $\Lang$-structures. If $M \cong N$, then for every $\varphi\of{x_1, \ldots, x_n} \in \Fml\of{\Lang}$ and every $a_1, \ldots, a_n \in \abs{M}$, we have that
    \begin{align*}
        M \models \varphi\!\brac{a_1, \ldots, a_n}
        \iff 
        N \models \varphi\!\brac{f\of{a_1}, \ldots, f\of{a_n}}
    \end{align*}
\end{boxtheorem}
\begin{proof}
    Fix a formula $\varphi\of{x_1, \ldots, x_n} \in \Fml\of{\Lang}$ with $n$ free variables. We prove the result by performing induction\footnote{That is, cases} on $\varphi$.
    
    Suppose $\varphi$ is atomic. Then, there are two cases.
    \begin{itemize}
        \item \underline{$\varphi\of{x_1, \ldots, x_n}$ is of the form $\tau_1\of{x_1, \ldots, x_n} = \tau_2\of{x_1, \ldots, x_n}$ for terms $\tau_, \tau_2 \in \Term\of{\Lang}$.}
        In this case, it is possible to show, by cases on what $\tau_1$ and $\tau_2$ can be (see \Cref{Ch1:Def:Term}) that $f$ is compatible with $\varphi$.

        \item \underline{$\varphi$ is of the form $R\of{x_1, \ldots, x_n}$ for some $R \in \Rels\of{\Lang}$.}
        This is true by definition of an isomorphism (see \Cref{Ch1:Def:Isomorphism}).
    \end{itemize}

    Suppose, now, that $\varphi$ is not atomic. It is enough to show that the result shows if $\varphi$ is of the form $\psi_1\of{x_1, \ldots, x_n} \land \psi_2\of{x_1, \ldots, x_n}$, $\psi_1\of{x_1, \ldots, x_n} \lor \psi_2\of{x_1, \ldots, x_n}$, $\neg \psi\of{x_1, \ldots, x_n}$, and $\forall x_1 \forall x_2 \ldots \forall x_{n} \psi\of{x_1, \ldots, x_n}$, as these would be adequate.

    \begin{itemize}
        \item \underline{$\varphi$ is of the form $\psi_1\of{x_1, \ldots, x_n} \land \psi_2\of{x_1, \ldots, x_n}$.}
        This is immediate from the definition of satisfaction: for all $a_1, \ldots, a_n \in \abs{M}$, we have that
        \begin{align*}
            M \models \varphi\!\brac{a_1, \ldots, a_n}
            & \iff
            M \models \psi_1\!\brac{a_1, \ldots, a_n}
            \text{ and }
            M \models \psi_2\!\brac{a_1, \ldots, a_n} \\
            & \iff
            N \models \psi_1\!\brac{f\of{a_1}, \ldots, f\of{a_n}}
            \text{ and }
            N \models \psi_2\!\brac{f\of{a_1}, \ldots, f\of{a_n}} \\
            &\iff
            N \models \varphi\!\brac{f\of{a_1}, \ldots, f\of{a_n}}
        \end{align*}
        as required.

        \item \underline{$\varphi$ is of the form $\psi_1\of{x_1, \ldots, x_n} \lor \psi_2\of{x_1, \ldots, x_n}$.}
        Similar.

        \item \underline{$\varphi$ is of the form $\neg \psi\of{x_1, \ldots, x_n}$.}
        \sorry

        \item \underline{$\varphi$ is of the form $\exists x_1, \psi\of{x_1, \ldots, x_n}$.}\footnote{This is enough because you can induct on the number of free variables, with exactly this being the inductive step.}
        Fix $a_1, \ldots, a_n \in \abs{M}$. Then,
        \begin{align*}
            M \models \varphi\!\brac{a_1, \ldots, a_n}
            &\iff
            M \models \exists x \varphi\!\brac{x, a_2, \ldots, a_n} \\
            &\iff
            \text{There is some } a \in \abs{M} \text{ such that } M \models \varphi\!\brac{b, a_2, \ldots, a_n} \\
            &\iff \text{There is some } a \in \abs{M} \text{ such that } N \models \varphi\!\brac{f\of{b}, f\of{a_2}, \ldots, f\of{a_n}} \\
            &\iff \text{There is some } b \in \abs{N} \text{ such that } N \models \varphi\!\brac{b, f\of{a_2}, \ldots, f\of{a_n}} \\
            &\iff N \models \exists y \varphi\!\brac{y, f\of{a_2}, \ldots, f\of{a_n}}
        \end{align*}
        where we note that the `$\implies$' direction of the fourth $\iff$ comes from taking $b = f(a)$ and the `$\impliedby$' direction comes from the fact that $f$ is surjective, meaning that we can take $a$ to be any element of $\abs{M}$ such that $f(a) = b$.
    \end{itemize}
    We can conclude by noting that the above cases are adequate. See \sorry. % Mention remark about adequacy.
\end{proof}

\begin{boxcorollary}
    Let $\Lang$ be a language and let $M$ and $N$ be $\Lang$-structures. If $M \cong N$, then $M \equiv N$.
\end{boxcorollary}
\begin{proof}
    Let $f : M \iso N$ be an isomorphism from $M$ to $N$. \sorry
    % We argue, by induction on sentences, that $M \models \varphi \iff N \models \varphi$ for all $\varphi \in \Sent\of{\Lang}$. \sorry % Need to do this - not done in 
\end{proof}

\begin{boxwarning}
    The notion of isomorphism is (much?) finer than elementary equivalence.
\end{boxwarning} % Add counter-example from Cori-Lascar

We end our discourse on first-order logic by briefly discussing the theory of deduction and proof.

\subsection{Deduction and Proof}

Let $\Lang$ be a language. Recall that $\Sent\of{\Lang}$ is the set of \textit{sentences} in $\Lang$. Throughout this subsection, fix a theory $T \subseteq \Sent\of{\Lang}$.

\begin{boxdefinition}[Provability]
    We say a sentence $\varphi \in \Sent\of{\Lang}$ is \textbf{provable from $T$}, denoted $T \vdash \varphi$, if there exists a sequence $\cycl{\varphi_1, \ldots, \varphi_n}$ of $\Lang$-sentences such that $\varphi_n = \varphi$ and for all $i < n$, either $\varphi_i \in T$ or $\varphi_{i}$ is obtained from $\cycl{\varphi_1, \ldots, \varphi_{i - 1}}$ via the standard deduction rules of first-order logic, namely, Modus Ponens and Generalisation.
\end{boxdefinition}

We can say something about what makes $T$ a ``sensible'' set from which to deduce things.

\begin{boxdefinition}[Consistency]
    We say that $T$ is \textbf{consistent} if there is no $\varphi \in \Sent\of{\Lang}$ such that $T \vdash \varphi$ and $T \vdash \neg\varphi$.
\end{boxdefinition}

Consistency is equivalent to model existence.

\begin{boxtheorem}[Gödel-Henkin]
    $T$ is consistent if and only if there is an $\Lang$-structure $M$ such that $M \models T$.
\end{boxtheorem}

We do not prove this theorem here, but we will make extensive use of it.

We end by recalling the compactness theorem for first-order logic.

\begin{boxtheorem}[Compactness, Gödel-Malcev]
    If for any finite $T_0 \subseteq T$, there is a 
\end{boxtheorem}

We now discuss the \textit{size} of a model and a theory.
\section{Cardinality and Categoricity}

Throughout this section, let $\Lang$ be a language.

\begin{boxdefinition}[Cardinality of a Structure]
    Let $M$ be an $\Lang$-structure. The \textbf{cardinality of $M$}, denoted $\norm{M}$, is the cardinality of its universe $\abs{M}$.
\end{boxdefinition}

We can also talk about the size of a theory.

\begin{boxdefinition}[Categoricity of a Theory]\label{Ch1:Def:Categoricity}
    Let $T$ be an $\Lang$-theory. Suppose $\lambda \geq \abs{L}$ is a cardinal. We say that \textbf{$T$ is $\lambda$-Categorical}, or that \textbf{$T$ is categorical in $\lambda$}, if for all $\Lang$-structures $M$ and $N$ such that $M, N \models T$ and $\norm{M} = \norm{N} = \lambda$, we have that $M \cong N$.
\end{boxdefinition}

Categoricity brings up interesting questions, such as the so-called \textit{spectrum problem}.

\subsection{The Spectrum Problem}

The spectrum of a theory with respect to a cardinal is defined as follows.

\begin{boxdefinition}[Spectrum]\label{Ch1:Def:Spectrum}
    Let $T$ be an $\Lang$-theory and let $\lambda$ be a cardinal. We define the \textbf{spectrum of $T$ with respect to $\lambda$} to be
    \begin{align*}
        I\of{\lambda, T} := \abs{
            \setst{\quotient{M}{\cong}}{M \models T \text{ and } \norm{M} = \lambda}
        }
    \end{align*}
    ie, $I\of{\lambda, T}$ denotes the number of isomorphism classes of models of $T$ of cardinality $\lambda$.
\end{boxdefinition}

It is obvious, from \Cref{Ch1:Def:Categoricity}, that a theory $T$ is $\lambda$-categorical if and only if $I\of{\lambda, T} = 1$. However, if $T$ is not $\lambda$-categorical, then it is, in general, quite difficult to compute $I\of{\lambda, T}$. In fact, for most theories and cardinals, computing the spectrum is an \textit{open problem}, referred to as the \textbf{spectrum problem}.

There has been some progress on this problem. Steinitz made the following determinations.

\begin{boxtheorem}[Steinitz]
    Let $\Lang$ be the language of fields, and let $T$ be the theory of algebraically closed fields of characteristic $p$ (obtained by adding the appropriate sentences to the Theory of Fields encountered in \Cref{Ch1:Eg:Thy_of_Fields}). Then,
    \begin{enumerate}[label = \normalfont \arabic*.]
        \item $I\of{\aleph_{0}, T} = \aleph_{0}$.
        \item For all $\lambda > \aleph_{0}$, $I\of{\lambda, T} = 1$.
    \end{enumerate}
\end{boxtheorem}

The spectrum problem has been worked on by some of the most eminent logicians of our time, including Rami's advisor, Saharon Shelah, who proved a famous conjecture by Morley (1965). More on the Spectrum Problem can be found on the associated \href{https://en.wikipedia.org/wiki/Spectrum_of_a_theory#:~:text=More%20precisely%2C%20for%20any%20complete,of%20a%20countable%20theory%20T.}{Wikipedia page}, and while this is not the most authoritative source, its contents are nonetheless interesting.

Morley also proved a famous conjecture by Łos from the 1950s, which since became known as Morley's Categoricity Theorem.

\begin{boxtheorem}[Morley's Categoricity Theorem, Morley 1965]
    Let $T$ be a theory in a language $\Lang$. Assume that $\abs{\Lang} \leq \aleph_{0}$. If $\exists \lambda > \aleph_{0}$ such that $T$ is $\lambda$-categorical, them $\forall \lambda > \aleph_{0}$, $T$ is $\lambda$-categorical.
\end{boxtheorem}

One of our objectives in this course is to prove Morley's Categoricity Theorem.

As a side note, Morley was initially a PhD student of Saunders MacLane's at the University of Chicago. Morley didn't initially finish his PhD, to the point of losing his stipend at Chicago, but somehow landed a job at Berkeley, where he proved this famous theorem. MacLane, a staunch category theorist, didn't believe Morley's work was quite enough to merit a PhD; nevertheless, after being persuaded by the then-nascent (and very excited) model theory community, he eventually relented and awarded Morley his degree.

Here, we end our discussion on the spectrum problem. Before proceeding further, we recall the basics of cardinal arithmetic.

\subsection{Cardinal Arithmetic}

We begin by introducing notation.

\begin{boxnotation}
    We denote by
    \begin{itemize}
        \item $\ZF$ the \ZFA\ of Set Theory
        \item $\AC$ the \ACA
        \item $\ZFC$ the \ZFCA
    \end{itemize}
\end{boxnotation}

We denote cardinality of a set $A$ by $\abs{A}$ or $\card{A}$ and write $\abs{A} = \abs{B}$ if and only if there is a bijection from $A$ to $B$. Informally, a \textbf{cardinal} is a measure of cardinality. That is, a set $\lambda$ is a cardinal if $\lambda = \abs{A}$ for some set $A$. We denote by $\aleph_{0}$ the cardinal of the natural numbers, which we will denote $\omega$ in any cardinal- or ordinal-theoretic context.

There are more precise ways in which we can define the notions of ordinals and cardinals. We do not do this here, but we mention that there is an appendix in Rami's book and several sections in my undergrad logic lecture notes\todo{Add references} that discuss this.

\begin{boxdefinition}[Cardinal Arithmetic]
    Let $\lambda, \mu$ be cardinals, with $\lambda = \abs{A}$ and $\mu = \abs{B}$. We denote
    \begin{align*}
        \lambda + \mu &:= \sorry \\
        \lambda \cdot \mu &:= \abs{A \times B}
    \end{align*}
\end{boxdefinition}

The following is a famed theorem of Tarski, a direct consequence of which is precisely the fundamental theorem of cardinal arithmetic.

\begin{boxtheorem}[Tarski]
    We can make the following deduction:
    \begin{align*}
        \ZF \vdash \parenth{\AC \lr \forall A, \, \abs{A} \geq \aleph_{0} \to \abs{A \times A} = \abs{A}}
    \end{align*}
    Equivalently,
    \begin{align*}
        \ZF \vdash \parenth{\AC \lr \forall A, \, \lambda \geq \aleph_{0} \to \lambda \cdot \lambda = \lambda}
    \end{align*}
\end{boxtheorem}

The fundamental theorem of cardinal arithmetic, which states that $\abs{\omega \times \omega} = \abs{\omega}$, is clearly just the specialisation of the above result to the case where $\lambda = \aleph_{0}$.

There is another fact that will be important for our purposes.

\begin{boxtheorem}
    For infinite cardinals $\lambda, \mu \geq \aleph_{0}$, we have
    \begin{align*}
        \lambda \cdot \mu = \max\of{\lambda, \mu} = \lambda + \mu
    \end{align*}
\end{boxtheorem}

The reason for discussing cardinal arithmetic is that we can exploit it to prove the existence of submodels of specific cardinalities.

\section{A Word on Submodels}

Fix a language $\Lang$.

\subsection{Submodel Existence}

We begin by defining the cardinality of a structure.

\begin{boxdefinition}[Cardinality of a Structure]
    Let $N$ be a $\Lang$-structure. We define $\card{N}$ to be the cardinality of the union of 
\end{boxdefinition}

We begin with the famed submodel theorem.

\begin{boxtheorem}[The Submodel Theorem {\cite[Theorem 2.8, pp. 49-50]{MOAB}}]
    Let $M$ be a $\Lang$-structure. Define $\lambda := \abs{\Lang} + \aleph_{0}$. If $A \leq \abs{M}$, then there exists a substructure $N \leq M$ such that
    \begin{enumerate}[label = (\alph*)]
        \item $\card{N} \geq A$
        \item $\card{\abs{N}} \leq \abs{A} + \lambda$
    \end{enumerate}
\end{boxtheorem}
\begin{proof}
    By recursion on $n < \omega$, define sets $\setst{B_{n} \ssq \abs{M}}{n < \omega}$ such that
    \begin{enumerate}
        \item $B_{0} = \setst{c \in \Consts^M}{c \text{ is a constant symbol of } c} \cup A$
        \item If $n < \omega$, then $\abs{B_n} \leq \abs{A} + \lambda$
        \item For all $n < \omega$, define $B_{n+1} := \setst{F^M\of{\overline{a}}}{\overline{a} \in B_{n}} \cup B_n$
    \end{enumerate}
    This is enough: if we have such a sequence of $B_n$, then we could take $B := \bigcup_{n < \omega} B_n$ and define $N := \cycl{B, F^M, R^M, C^M}$. We can show that this satisfies the desired conditions.
    \begin{enumerate}[label = (\alph*)]
        \item \sorry
        \item \sorry
    \end{enumerate}

    Given these, all that remains now is to show that this is possible. \sorry\todo{Finish using textbook proof}
\end{proof}

\subsection{Elementary Submodels}

Recall the definition of elementary substructures (\sorry). In this subsection, we define an analogous notion for models.

\begin{boxdefinition}[Elementary Submodels]
    Let $M, N$ be $\Lang$-structures. We say that \textbf{$M$ is an elementary submodel of $N$}, denoted $M \preceq N$, if
    \begin{enumerate}
        \item $M \subseteq N$
        \item $M \models \varphi\!\brac{a_1, \ldots, a_n}$ iff $N \models \varphi\!\brac{a_1, \ldots, a_n}$ for every $\varphi \in \Fml\of{\Lang}$ and $a_1, \ldots, a_n \in \abs{M}$.
    \end{enumerate}
\end{boxdefinition}

We can relate this to the notion of elementary substructures in the following manner.

\begin{boxtheorem}[Tarski-Vaught 1956]
    Let $M, N$ be $\Lang$-structures with $M \subseteq N$. If $M \preceq N$, then $M \equiv N$.
\end{boxtheorem}

The converse is not true.

\begin{boxcexample}
    \sorry
    % If $M \equiv N$, then it is not necessary that $M \preceq N$.
\end{boxcexample}

\subsection{Chains of Substructures}

Throughout this subsection, fix a linear order $\parenth{\I, \le}$.

\begin{boxdefinition}[Chain]
    Let $\setst{M_{i}}{i \in \I}$ be $\Lang$-structures. We say they form a \textbf{chain} if for all $i_1, i_2 \in \I$, if $i_1 < i_2$ then $M_{i_1} \subseteq M_{i_2}$.
\end{boxdefinition}

For the remainder of this subsection, fix a chain $\setst{M_{i}}{i \in \I}$. It is easy to show that
\begin{align*}
    N := \bigcup_{i \in \I} M_i
\end{align*}
is an $\Lang$-structure as well. Moreover, $M_i \subseteq N$ for all $i \in \I$.

We can ask ourselves a natural question: suppose $T$ is a theory in $\Lang$ and that $\forall i \in \I$, $M_i \models T$. Is it necessarily true that $N \models T$ as well?

The answer turns out to be no when $\abs{I} \geq \aleph_0$.

\begin{boxcexample}
    Take $I = \omega$ and 
\end{boxcexample}

\subsection{The Löwenheim-Skolem Theorems}
\section{A Word on Submodels}

Fix a language $\Lang$.

\subsection{Submodel Existence}

We begin by defining the cardinality of a structure.

\begin{boxdefinition}[Cardinality of a Structure]
    Let $N$ be a $\Lang$-structure. We define $\card{N}$ to be the cardinality of the union of $\textbf{F}^N(\Lang)\cup \textbf{C}^N(\Lang)\cup \textbf{R}^N(\Lang)\cup |N|$.
\end{boxdefinition}

We begin with the famed submodel theorem.

\begin{boxtheorem}[The Submodel Theorem {\cite[Theorem 2.8, pp. 49-50]{MOAB}}]\label{Ch1:Thm:Submodel_Thm}
    Let $M$ be a $\Lang$-structure. Define $\lambda := \abs{\Lang} + \aleph_{0}$. If $A \leq \abs{M}$, then there exists a substructure $N \leq M$ such that
    \begin{enumerate}[label = (\alph*)]
        \item $\card{N} \geq A$
        \item $\card{\abs{N}} \leq \abs{A} + \lambda$
    \end{enumerate}
\end{boxtheorem}
\begin{proof}
    By recursion on $n < \omega$, define sets $\setst{B_{n} \ssq \abs{M}}{n < \omega}$ such that
    \begin{enumerate}
        \item $B_{0} = \setst{c \in \Consts^M}{c \text{ is a constant symbol of } c} \cup A$
        \item If $n < \omega$, then $\abs{B_n} \leq \abs{A} + \lambda$
        \item For all $n < \omega$, define $B_{n+1} := \setst{F^M\of{\overline{a}}}{\overline{a} \in B_{n}} \cup B_n$
    \end{enumerate}
    This is enough: if we have such a sequence of $B_n$, then we could take $B := \bigcup_{n < \omega} B_n$ and define $N := \cycl{B, F^M, R^M, C^M}$. We can show that this satisfies the desired conditions.
    \begin{enumerate}[label = (\alph*)]
        \item \sorry
        \item \sorry
    \end{enumerate}

    Given these, all that remains now is to show that this is possible. \sorry\todo{Finish using textbook proof}
\end{proof}

\subsection{Elementary Submodels}

Recall the definition of elementary substructures (\sorry). In this subsection, we define an analogous notion for models.

\begin{boxdefinition}[Elementary Submodels]
    Let $M, N$ be $\Lang$-structures. We say that \textbf{$M$ is an elementary submodel of $N$}, denoted $M \preceq N$, if
    \begin{enumerate}
        \item $M \subseteq N$
        \item $M \models \varphi\!\brac{a_1, \ldots, a_n}$ iff $N \models \varphi\!\brac{a_1, \ldots, a_n}$ for every $\varphi \in \Fml\of{\Lang}$ and $a_1, \ldots, a_n \in \abs{M}$.
    \end{enumerate}
\end{boxdefinition}

We can relate this to the notion of elementary substructures in the following manner.

\begin{boxtheorem}[Tarski-Vaught 1956]
    Let $M, N$ be $\Lang$-structures with $M \subseteq N$. If $M \preceq N$, then $M \equiv N$.
\end{boxtheorem}

The converse is not true.

\begin{boxcexample}
    \sorry
    % If $M \equiv N$, then it is not necessary that $M \preceq N$.
\end{boxcexample}

\subsection{The Tarski-Vaught Test}

In this subsection, we explore a monumental result by Tarski and Vaught that gives a sufficient and necessary condition for a substructure to be elementary.

We begin by introducing some notation.

\begin{boxlnotation}
    Denote by $\star_{\psi}$ the statement
    \begin{align}
        \text{For all } a_1, \ldots, a_n \in \abs{M},
        \qquad
        M \models \psi\!\brac{a_1, \ldots, a_n}
        \iff
        N \models \psi\!\brac{a_1, \ldots, a_n}
    \end{align}
    for some $\psi \in \Fml(L)$.
\end{boxlnotation}

Next, we note a fact about substructures.

\begin{boxlemma}\label{Ch1:Lemma:Tarski-Vaught-star-psi}
    Let $N$ be an $L$-structure and let $M \ssq N$ be a substructure of $N$. Then, $\star_{\psi}$ holds for all quantifier-free formulae $\psi \in \Fml(L)$.
\end{boxlemma}
\begin{proof}
    \sorry
\end{proof}

\begin{boxtheorem}[The Tarski-Vaught Test]\label{Ch1:Thm:Tarski-Vaught}
    Let $M, N$ be $L$-structures with $M \ssq N$. Then, the following are equivalent.
    \begin{enumerate}
        \item $M \preceq N$
        \item If, for every $\varphi\of{y, x_1, \ldots, x_n} \in \Fml(L)$ and $a_1, \ldots, a_n \in \abs{M}$,
        \begin{align*}
            N \models \exists y \, \varphi\of{y ,a_1, \ldots, a_n}
        \end{align*}
        then there is some $b \in \abs{M}$ such that $N \models \varphi\!\brac{b, a_1, \ldots, a_n}$
    \end{enumerate}
\end{boxtheorem}
\begin{remark}
    We can see this as 'a more ``algebraic'' notion of being a submodel.' What is a formula? A list of quantifiers, connectives, etc. - for example, we can think of polynomials in several variables, which we wish to solve. If $\vp(y,x)$ is a set of finitely many equations (which we wish to solve), we can see this result as telling us that if there exists a solution to the system $y\in N$, there is also a $b$ in the substructure $M$ which also solves the same system. 
\end{remark}
\begin{proof}[Proof of \Cref{Ch1:Thm:Tarski-Vaught}.]
    \begin{description}
        \item[\underline{$1 \implies 2$.}]
        Fix  $\varphi\of{y, x_1, \ldots, x_n} \in \Fml(L)$ and $a_1, \ldots, a_n \in \abs{M}$. Suppose
        \begin{align*}
            N \models \exists y \, \varphi\of{y ,a_1, \ldots, a_n}
        \end{align*}
        Since $M \preceq N$, by definition of satisfaction, we know that
        \begin{align*}
            M \models \exists y \, \varphi\of{y ,a_1, \ldots, a_n}
        \end{align*}
        This tells us that there is $b \in \abs{M}$ witnessing $\varphi$, meaning that
        \begin{align*}
            M \models \exists y \, \varphi\of{b ,a_1, \ldots, a_n}
        \end{align*}
        Then, since $M \preceq N$, we have that
        \begin{align*}
            N \models \exists y \, \varphi\of{b ,a_1, \ldots, a_n}
        \end{align*}
        as required.

        \item[\underline{$2 \implies 1$.}]
        We show, by induction on $\varphi\of{y, x_1, \ldots, x_n} \in \Fml(L)$, that $\star_{\psi}$ holds for all $\psi \in \Fml(L)$. Recall, from \Cref{Ch1:Lemma:Tarski-Vaught-star-psi}, that $\star_{\psi}$ does hold for quantifier-free formulae $\psi$. In particular, it holds for atomic formulae. We can now consider the different possible cases on $\psi$.
        \begin{enumerate}
            \item \underline{$\psi$ is of the form $\psi_1 \land \psi_2$.} \newline
            \sorry

            \item \underline{$\psi\of{x_1, \ldots, x_n}$ is of the form $\exists y \, \varphi\of{y, x_1, \ldots, x_n}$.} \newline
            Assume that $M \models \psi\!\brac{a_1, \ldots, a_n}$. Then, by assumption, there is some $b \in \abs{M}$ such that $M \models \varphi\!\brac{b, a_1, \ldots, a_n}$. Then, $N \models \varphi\!\brac{b, a_1, \ldots, a_n}$ because $b \in \abs{M} \ssq \abs{N}$. Thus,
            \begin{align*}
                N \models \exists y \, \varphi\of{y, a_1, \ldots, a_n}
            \end{align*}
            ie,
            \begin{align*}
                N \models \psi\!\brac{a_1, \ldots, a_n}
            \end{align*}
        \end{enumerate}
    \end{description}
    By $\star_{\psi}$, $b \in \abs{M}$ % The truth is I'm a bit confused and a little lost - need to think a little about the reasoning here. Should be ok but need to think about it a bit
\end{proof} %...not alone (heh :)) - there's probably time (which will be made) to do just that :D Sounds good

\subsection{Chains of Substructures}

Throughout this subsection, fix a linear order $\parenth{\I, \le}$.

\begin{boxdefinition}[Chain]
    Let $\setst{M_{i}}{i \in \I}$ be $\Lang$-structures. We say they form a \textbf{chain} if for all $i_1, i_2 \in \I$, if $i_1 < i_2$ then $M_{i_1} \subseteq M_{i_2}$.
\end{boxdefinition}

\begin{boxlnotation}
    For the remainder of this subsection, fix a chain $\setst{M_{i}}{i \in \I}$. Define
    \begin{align*}
        N := \bigcup_{i \in \I} M_i
    \end{align*}
\end{boxlnotation}
It is easy to show that $N$ is an $\Lang$-structure as well. Moreover, $M_i \subseteq N$ for all $i \in \I$.

We can ask ourselves a natural question: suppose $T$ is a theory in $\Lang$ and that $\forall i \in \I$, $M_i \models T$. Is it necessarily true that $N \models T$ as well?

The answer turns out to be no when $\abs{I} \geq \aleph_0$.

\begin{boxcexample}
    Take $I = \omega$ and \sorry
\end{boxcexample}

We can instead define elementary chains, which are the analogues of chains for elementary substructures.

%just not the elementary part? yeah you're right (yes, we do have chains)
\begin{boxdefinition}[Elementary Chain] % I got this dw
    We say that $\setst{M_{i}}{i \in \I}$ form an \textbf{elementary chain} if for all $i_1, i_2 \in \I$, if $i_1 < i_2$ then $M_{i_1} \preceq M_{i_2}$.
\end{boxdefinition}

We can apply \Cref{Ch1:Thm:Tarski-Vaught} to prove an important result on elementary chains.

\begin{boxtheorem}[Tarski-Vaught Chain Theorem]
    Assume $\setst{M_i}{i \in I}$ is an elementary chain. Let $N$ be an $L$-structure. Then, the $L$-structure
    \begin{align*}
        N = \bigcup_{i \in I} M_i
    \end{align*}
    satisfies the property that for all $i \in I$, $M_i \preceq N$.
\end{boxtheorem}
\begin{proof}
    Since we already know that $M_i \ssq N$, it is enough to show that for every $\psi\of{x_1, \ldots, x_n} \in \Fml(L)$ and every $i \in I$, $\star_{\psi}$ holds (where $\star_{\psi}$ is the formula defined as local notation in the previous subsection, considered along with the substructure $M_i$ of $N$).

    We know that for all $i \in I$, since $M_i \ssq N$, $\star_{\psi}$ holds for atomic formula. We induct on logical connectives and quantifiers to exhaustively prove that the statement $\forall i \in I, \star_{\psi}$ is true. We only do a few cases explicitly.

    \begin{enumerate}
        \item \underline{$\psi\of{x_1, \ldots, x_n}$ is of the form $\neg \vp\of{x_1, \ldots, x_n}$.}

        Then, for all $i \in I$, $M_i \models \psi\!\brac{a_1, \ldots, a_n}$ iff $M_i \not\models \vp\!\brac{a_1, \ldots, a_n}$. By the induction hypothesis that $\forall i \in I, \star_{\phi}$ holds for all formulae $\phi$ with fewer quantifiers than $\psi$, we can conclude that
        \begin{align*}
            M_i \not\models \vp\!\brac{a_1, \ldots, a_n}
            \iff
            N \not\models \vp\!\brac{a_1, \ldots, a_n}
        \end{align*}
        This tells us that $N \models \psi\!\brac{a_1, \ldots, a_n}$ for every interpretation $a_i$ of $x_i$.

        \item \underline{$\psi\of{x_1, \ldots, x_n}$ is of the form $\exists y\,  \vp\of{y, x_1, \ldots, x_n}$.}

        Fix $i \in I$. Then, $M_i \models \psi\!\brac{a_1, \ldots, a_n} \implies M_i \models \exists y \, \varphi\!\of{y, a_1, \ldots, a_n}$. Then, there is some $b \in \abs{M_i}$ such that $M_i \models \varphi\!\brac{b, a_1, \ldots, a_n}$. This tells us, by the induction hypothesis, that $N \models \varphi\!\brac{b, a_1, \ldots, a_n}$ for some $b \in \abs{M_i} \ssq \abs{N}$. Thus, $N \models \exists y\,  \vp\of{y, x_1, \ldots, x_n}$, meaning $N \models \psi\!\brac{a_1, \ldots, a_n}$ as required.
    \end{enumerate}
    We can argue similarly for other quantifiers and connectives.
\end{proof}

\begin{boxcorollary}
    If $T$ is an $L$-theory, then if $M_i \models T$ for all $i \in I$ then $N \models T$ as well.
\end{boxcorollary}

We don't prove this corollary here.

We finally mention an additional nuance. Suppose $i \in I$ satisfies $a_1, \ldots, a_n \in \abs{M_i}$ and $N \models \psi\!\brac{a_1, \ldots, a_n}$. Say that $\psi$ is of the form $\exists y \, \varphi\of{y, x_1, \ldots, x_n}$. Then, by the definition of satisfaction, we know that there is some $b \in \abs{N}$

\begin{boxdefinition}[directed poset]
    Let $(I, <)$ be a poset, we say that $I$ is ``directed" if $\forall i, j\in I$ there exists $b\in I$ such that $i\leq k, j\leq k$.
\end{boxdefinition}
\begin{remark}
    This theorem can be extended - we don't need a linearly ordered set, possibly a directed poset would suffice? (We won't see this in this course.)
\end{remark}

%this is why elementary submodel is stronger than having the same theory(?) (just getting all the terms straight in my head, sorry again) - alright, no worries (I will continue doing that)
% Hmm - I might have them mixed up asw, need to look at it again. (I want to say yes) - Lmk once you figure it out!
% Other way round - sentences are formulae without free vars - sure

\subsection{Restrictions and Expansions} % Move to section on submodels!!!!

Before going any further, we will need to define a central tool: restrictions and expansions. Throughout this subsection, fix a language $L$ and an $L$-structure $M$.

\begin{boxdefinition}[Restriction/Expansion]\label{Ch1:Def:Res_Exp}
    Let $L_1 \subseteq L$, so that $L$ contains relations, functions, and constants $\Rels\of{L_1}$, $\Funcs\of{L_1}$, and $\Consts\of{L_1}$. The \textbf{restriction of $M$ to $L$} is the $L_1$-structure
    \begin{align*}
        M \vert_{L_1} := \cycl{\abs{M}, \Rels^{M}\of{L_1}, \Funcs^M\of{L_1}, \Consts^M\of{L_1}}
    \end{align*}
    Dually, we say that \textbf{$M$ is the expansion of $M\vert_{L_1}$ to $L$}.
\end{boxdefinition}

A good example of this is to model-theoretically encode the fact that every field is also an abelian group (additively).

\begin{boxexample}[Restriction: Fields to Abelian Groups]
    Let $L$ be the language of fields and let $L_1$ be the language of (additively expressed) abelian groups. Then, if
    \begin{align*}
        M = \cycl{\Q, +, \cdot, 0, 1}
    \end{align*}
    then its restriction to $L_1$ is
    \begin{align*}
        M\vert_{L_1} = \cycl{\Q, +, 0}
    \end{align*}
\end{boxexample}

\subsection{The Löwenheim-Skolem Theorems}

Throughout, let $L$ be a language.

We begin with the downwards theorem, which gives us substructures with control over cardinality. It looks similar to the Submodel Theorem (\Cref{Ch1:Thm:Submodel_Thm}).

\begin{boxtheorem}[Downwards Löwenheim-Skolem-Tarski Theorem]\label{Ch1:Thm:Downwards_LS}
    Let $M$ be an $L$-structure. Define $\lambda := \abs{L} + \aleph_0$. For all $A \ssq \abs{M}$, there is some $N \preceq M$ with $\abs{N} \supseteq A$ and $\abs{\abs{N}} \leq \abs{A} + \lambda$, where $\abs{\abs{N}}$ refers to the cardinality of the universe of $N$.
\end{boxtheorem}
Recall that \Cref{Ch1:Thm:Submodel_Thm} gives us the existence of $N \leq M$ with the desired properties. The difference is that in \Cref{Ch1:Thm:Downwards_LS}, we have elementarity.
\begin{proof}[Proof of \Cref{Ch1:Thm:Downwards_LS}.]
    Fix $A \ssq \abs{M}$. Fix $\varphi\of{y, x_1, \ldots, x_n} \in \Fml(L)$. Fix a well-ordering $\leq$ of $\abs{M}$, which we pick using the Axiom of Choice.
    
    We define the function $G_{\varphi} : \abs{M} \times \cdots \times \abs{M} \to \abs{M}$ as follows: for all $b_1, \ldots, b_n$, define
    \begin{align}
        G_{\varphi}\of{b_1, \ldots, b_n}
        =
        \begin{cases}
            \min_{\leq}\!\abs{M} & \text{ if } M \not\models \exists y \, \varphi\of{y, x_1, \ldots, x_n} \\
            \min_{\leq}\!\setst{a \in \abs{M}}{M \models \varphi\!\brac{a, b_1, \ldots, b_n}} & \text{ if } M \models \exists y \, \varphi\of{y, x_1, \ldots, x_n}
        \end{cases}
        \label{Ch1:Eq:DLS_Skolem_Fn}
    \end{align}
    Thus, for every $b_1, \ldots, b_n \in \abs{M}$, $G_{\varphi}$ gives the least element $a \in \abs{M}$ such that $\varphi\!\brac{a, b_1, \ldots, b_n}$ is satisfied (and returns a junk value of there is no such element).

    We now augment our language $L$ in the following manner. Define
    \begin{align}
        L_1 := L \cup \setst{G_{\varphi}\of{x_1, \ldots, x_n}}{\varphi\of{y, x_1, \ldots, x_n} \in \Fml(L)}
        \label{Ch1:Eq:DLS_Lang_Augmentation}
    \end{align}
    That is, we create $L_1$ by adding to $L$ all the constant symbols that correspond to \sorry.

    Let $M_1$ be the expansion of $M$ to $L$ (cf. \Cref{Ch1:Def:Res_Exp}). Observe that
    \begin{align*}
        \abs{L_1} \leq \abs{L} + \abs{\Fml(L)} \leq \lambda + \aleph_0 \cdot \lambda = \lambda
    \end{align*}
    Now, we can apply the Submodel Theorem (\Cref{Ch1:Thm:Submodel_Thm}) to $M_1$ and $A$ to obtain some $N_1 \ssq M_1$ (as $L_1$-structures) such that $\abs{N_1} \supseteq A$ and $\abs{\abs{N}} \leq \abs{A} + \lambda$.

    Define $N := N_1\vert_{L}$, the restriction of $N_1$ to $L$. To show that $N$ has the properties desire, we really only need to show that $N\vert_L \preceq M \vert_L$. We do this by using the Tarski-Vaught test (\Cref{Ch1:Thm:Tarski-Vaught}) to prove that $M_1\vert_{L} \preceq N_1\vert_{L}$.

    Fix $\psi\of{y, x_1, \ldots, x_n} \in \Fml(L)$. Suppose that $M \models \exists y \, \psi\of{y, a_1, \ldots, a_n}$ for some $a_1, \ldots, a_n \in \abs{N_1}$. Then, by definition of $G$ for formulae, we have that $G_{\psi}\of{a_1, \ldots, a_n} \in \abs{M_1}$ is the smallest $b \in \abs{M_1}$ such that $M_1 \models \psi\!\brac{b, a_1, \ldots, a_n}$. Since $a_1, \ldots, a_n \in \abs{N_1}$, and since $N$ is closed under taking $G_{\psi}$, we must have $b \in N_1$.
    
    % This is so entertaining fr
    %very - also good to know that's one way to get a PhD - we'll see (also sorry, thank you for scribing)
    % No problem - This is a not unusual but weird-looking (to me) argument, seems somewhat... not artificial, but... not quite sure what word to use...
    % Haha
\end{proof}

We will find that the techniques of defining functions like $G_{\varphi}$ in \eqref{Ch1:Eq:DLS_Skolem_Fn} and augmenting languages as in \eqref{Ch1:Eq:DLS_Lang_Augmentation} will come up time and time again in model theory.

\begin{boxtheorem}[Upwards Löwenheim-Skolem Theorem]
    Suppose $T$ has an infinite model. Then given any cardinal $\lambda \geq \aleph_0+|\textbf{L}(T)|$ there exists some $L$-structure $M$ of cardinality $||M||=\lambda$ and $M\models T$.
\end{boxtheorem}

\begin{proof}
    We will just add $\lambda$ many constants to a model of $T$. In particular, we let $T^k:=T\cup\{c_i\neq c_j|i\neq j<\lambda\}$. By the compactness theorem $T^k$ has a model $N$, and if we now let $A=\{c_i^M| i<\lambda\}$ then applying Downward Lowenheim-Skolem Tarski we can obtain some $M<N\restriction \textbf{L}(T)$ of cardinality $\lambda$ (completing the proof).
\end{proof}

\begin{boxdefinition}[Complete Diagram on $M$]
    Let $M$ be an $L$-structure and $L_M:=\textbf{L}(M)\cup \{c_a|a\in M\}$. Letting $M':=\{ M, c_a\}_{a\in |M|}$ with the constants interpreted such that $c_a^{M'}=a$ for all $a\in |M|$, we define $CD(M)=Th(M')$ denote the ``complete diagram of $M$.''
 \end{boxdefinition}

 \begin{boxdefinition}[Elementary Diagram of $M$] 
     We now define $$ED(M)=\{\vp \in CD(M)|\vp\text{ is quantifier free}\}$$
 \end{boxdefinition}

 \begin{boxlemma}[Lemma 1]
     Suppose $N\models ED(M)$ and let $N^*:=N\restriction L$. Then there exists $\vp:|M|\rightarrow |N^*|$ an injective homomorphism (not necessarily injective).
 \end{boxlemma}
 For a proof, we can just take $\vp(a)=c_a^{N^*}$.

 \begin{boxlemma}[Lemma 2]
     Suppose $N\models CD(M)$ with $N^*:= N\restriction L$. The there exists $\vp:|M|\rightarrow |N|$ an elementary embedding [i.e. $\vp[M]\prec N$].
 \end{boxlemma}

 As an application, given $M$ and some $\Gamma$ a set of sentences, if $CD(M)\cup \Gamma$ which is consistent, it follows that there must exist some $N$ such that $M$ is an elementary submodel of $N$ and $N\models \Gamma$.

 \begin{boxtheorem}[Upward Lowenheim-Skolem-Tarski]
     Given an infinite $L$-structure $M$ and any cardinal $\lambda\geq ||M||+|\textbf{L}(M)|$ there exists some $N>M$ of cardinality $\lambda$.
 \end{boxtheorem}
 For this one-line proof, just apply \sorry to $CD(M)$.

 What are the basic theorems of model theory? ULS, DLST, and Compactness Theorem - and it turn out that these are equivalent to AC in ZFC!

\begin{remark}
     A fact proved by Lauchli and Levi is that $ZF\vdash [CT\leftrightarrow BPI]$ where $BPI$ deals with ``Boolean Prime Ideals'' \sorry (see book), and further in $ZF$ all of this is equivalent to Tychonoff's Theorem. The effect of this is that it's very difficult to do interesting model theory without choice.
\end{remark}
\begin{remark}
    ``I feel it in my bones, it is true.'' - Shelah's response to (Grossberg's) the question of how one can feel comfortable using Axiom of Choice in Model Theory. Made quite an impression on Professor Grossberg.
\end{remark}

\section{Models of Peano Arithmetic}

In this section, we build many models of Peano arithmetic, the most standard and universal of which is $\omega$, the natural numbers.

\subsection{The Language and Theory of Peano Arithmetic}

In this subsection, we set up our study of Peano arithmetic by defining the language in which we will work and the theory we will seek to model. We begin with some notation.

\begin{boxnotation}
    Let $\PA$ denote the theory of Peano arithmetic expressed in a language $L$, the `language of Peano arithmetic'.
\end{boxnotation}

\sorry

\subsection{Existence of Non-Standard Models of Peano Arithmetic}

We know that there is a standard model of Peano arithmetic, denoted $N$. This consists intuitively of the natural numbers---that is, ordinals less than $\omega$---with the usual addition, multiplication, additive and multiplicative identities. It turns out there are also other models, whose existence we will see in this subsection.

Before that, we mention that any model of Peano arithmetic admits a linear order.

\begin{boxtheorem}
    The following is a valid deduction.
    \begin{align*}
        \PA \vdash \forall x \forall y \exists z \parenth{y = x + z}
    \end{align*}
\end{boxtheorem}

\begin{boxtheorem}
    There exists some $M\models \PA$ such that $M$ is not isomorphic to the standard model $N$.
\end{boxtheorem}
\begin{proof}
    Augment the language $L$ to the $L_1 = L_{\PA} \cap \set{c}$ by adding a new constant symbol $c$. Let
    \begin{align*}
        T_1 := \PA \cup \set{c \neq 0} \cup \setst{c \neq \underbrace{1 + \cdots + 1}_{n \text{ times}}}{n < \omega}
    \end{align*}
    We show, using the Compactness Theorem (\sorry), that there is a model $M$ of $T_1$. Given $T_0 \subseteq T_1$ finite, let $n_0 < \omega$ be the largest natural number such that
    \begin{align*}
        c \neq \underbrace{1 + \cdots + 1}_{n \text{ times}}
    \end{align*}
    lies in $T_0$. Define the $L_1$-structure
    \begin{align*}
        M_0 := \cycl{\omega, +, \cdot, 0, 1, a}
    \end{align*}
    be the expansion of $N$ to $L_1$., with $c^{M_0} = a$. Then, $M_0 \models T_0$. Hence, since every finite subset of $T_1$ has a model, so does $T_1$. Call this model $M_1$. Denote by $M$ the restriction of $M_1$ to $L_{\PA}$.

    Now that we have established the existence of $M$, all that remains is to show that $M$ is not isomorphic to $N$. Let $b = c^M$. Suppose $M \cong N$ via an isomorphism $f$. Then, $f(b) \in \omega$, so there is some $n < \omega$ such that
    \begin{align*}
        f(b) = \underbrace{1 + \cdots + 1}_{n \text{ times}}
    \end{align*}
   
\end{proof} % I'm slightly unsure about interpreting c in M as b, because c is a symbol in the larger language L_1, not L... I get what you're going for (makes a lot of sense) but I'm stuck on the previous step - defining b. Will maybe ask Rami after class

%asking would be good - as I understand, all we need for the prooof is that there is an interpretation of c in M (c arises as a constant in the larger language, but nonetheless the existence of an interpretation implies that such an element exists in M) - I think now he's working on something else

% Makes sense. Yeah this might be something else

%we can check after

% (Unless you know?)
%(lmao)
% Oh wait. His proof isn't over

\subsection{Famous Results on Peano Arithmetic}

Recall the definition of the \textit{spectrum of a theory} (\Cref{Ch1:Def:Spectrum}). It turns out, we can use the idea of a spectrum to say something rather sophisticated about countable models of Peano Arithmetic.

\begin{boxtheorem}\label{Ch1:Thm:Spectrum_of_PA}
    $I\of{\aleph_0, \PA} = 2^{\aleph_0}$.
\end{boxtheorem}

We know that $\PA$ admits a model (indeed, a standard model). Therefore, we know that $\PA$ is a consistent theory. However, Gödel famously showed that it is not possible to prove consistency of $\PA$ using $\PA$ alone. That is, he proved the following.

\begin{boxtheorem}[Gödel's Second Incompleteness Theorem]
    $\PA \not\vdash \PA \text{ is consistent}$
\end{boxtheorem}
\begin{remark}
    Note that no number theorist would accept this as an example of a statement of interest in number theory which is not provable in $\PA$ (so says the model theorist). 
\end{remark}

Another interesting impossibility result involves Ramsey theorem.

\begin{boxtheorem}[Paris-Harrington 1976]
    $\PA \not\vdash \text{A special case of Ramsey's Theorem}$.
\end{boxtheorem}

\subsection{True Arithmetic and the Twin Prime Conjecture}

Recall the definition of the theory of a model (\Cref{Ch1:Def:Thy_of_Model}). Since $\cycl{\omega, +, \cdot, 0, 1}$ is obviously a model of Peano Arithmetic, its \textit{theory} strictly contains the theory of Peano Arithmetic. We call this new theory the theory of true arithmetic.

\begin{boxdefinition}[True Arithmetic]\label{Ch1:Def:True_Arith}
    Define the \textbf{theory of true arithmetic}, denoted $\TA$, to be
    \begin{align*}
        \TA := \Th\of{\cycl{\omega, +, \cdot, 0, 1}} \supsetneq \PA
    \end{align*}
\end{boxdefinition}

The following is known about the spectrum of true arithmetic.

\begin{boxtheorem}\label{Ch1:Thm:Spectrum_of_TA}
    For all cardinals $\lambda \geq \aleph_0$, $I\of{\lambda, \TA} = 2^{\lambda}$.
\end{boxtheorem}

Indeed, taking $\lambda = \abs{\omega} = \aleph_0$ gives us something reminiscent of \Cref{Ch1:Thm:Spectrum_of_PA}.

For the remainder of this subsection, we will talk about how we can use the theory of true arithmetic to study the twin prime conjecture. We begin with notation.

\begin{boxlnotation}
    Denote by $\psi$ the formula in the language of true arithmetic expressing that there are infinitely many twin primes. Let $\calP$ denote the set of prime numbers, a subset of $\omega$. We will also use the symbol $\mid$ in an in-fix manner to denote
    \begin{align*}
        x \mid y \iff \exists x \brac{y = z \cdot x}
    \end{align*}
\end{boxlnotation}

\begin{boxlemma}
    For every $S \ssq \calP$, there is a countable model $M_S \models \TA$. Moreover, $\exists a_S \in \abs{M_S}$ such that for all $p \in \calP$, $M_S \models p \mid a_S$ if and only if $p \in S$.
\end{boxlemma}
\begin{proof}
    $I(\aleph_0, TA)=2^{\aleph_0}$ there exists $\mu<2^{\aleph_0}$ infinite such that $\exists \{M_i|i<\mu\}$ a sequence of countable structures a;; models of TA such that \textbf{if} $M\models TA$ countable \textbf{then}  $\exists n<\mu$ such that $M\cong M_i$. Let $L'=L_{PA}\cup \{c\}$ where $c$ is a constant. Given the assumption of our lemma, there exists $\{N_{\alpha}|\alpha<2^{\aleph_0}\}$ all countable $L'$-structures such that $N_{\alpha}\models TA$ and $\alpha \neq \beta \implies N_{\alpha}\ncong N_{\beta}$. 

    [We can let $M'_s:=\la M_s, a_s\ra$ denote the expansion of $M_s$ to $L'$. For $S_1\neq S_2$ we note there exists $q\in S_1, q \notin S_2$. For such $q$ we will thus have $M'_{s_1}\models q\mid c$ but $M'_{s_2}\models \neq (g\mid c)$, from which we conclude that $M'_{s_1}, M'_{s_2}$ cannot be elementarily equivalent.]

    Since $N_2\models TA$ we will have $\la N_{\alpha}\rest L_{PA}/\cong|\alpha <2^{\aleph_0}\}\ssq \{M_i|i<\mu\}$ but then we have $\{N_{\alpha}/\cong |\alpha<2^{\aleph_0}\}\ssq \{ (M_i, a)/\cong|a\in |M_i|, i<\mu\}$. Considering the cardinalities of these sets we will have $\sum_{i<\mu}||M_i||\leq \mu \aleph_0=\mu$ but $2^{\aleph_0}\leq \mu$ contradicts $\mu<2^{\aleph_0}$. \sorry

    (if we don't have continuum many we can write down the countable subsets indexed by $\mu$, adding a countable single we will get continuum many isomorphism types of expanded language, but how many interpretations of constant? countable. This is a kind of approximation

    
     (take $a_s$ interpreting the constant)

    (taking 2 diff sets, prime in one will not be in other)
\end{proof}

\subsection*{class 9-17 (move where you please)}
\begin{boxdefinition}[Ordered Field]
    We define an ordered field $(F, +, \cdot, 0, 1, \leq)$ as satisfying the sentence $\forall x \forall y \forall z[x\leq y\rightarrow x+z<y+z]$ and $\forall x \forall y[x<y\rightarrow \forall z[z>0\rightarrow x\cdot z<y\cdot z]]]$
\end{boxdefinition}
\begin{boxdefinition}[Archimedean Ordered Field]
    An ordered field is ``Archimedean'' if it satisfies $\phi=\begin{cases}
    \forall a \in F\text{ if }a>0\text{ then there is integer }n\text{such that }a_0+\ldots + a_n\geq 1\end{cases}$
\end{boxdefinition}

For example, $\R$ is an Archimedean ordered field.

One may ask if there is some finite set of sentences $T$ in the language $\langle +, \cdot, 0, 1, \leq\rangle$ equivalent to $\phi$ - the answer is no:

\begin{boxtheorem}
    There exists an extension $M>(\R, + \cdot, 0, 1, \leq)$ which is not Archimedean.
\end{boxtheorem}
\begin{proof}
    We let $c$ denote a new constant and then define $T^*:=CD(\R)\cup \{c>0\}\cup\{n\cdot c<1|n<\omega\}$. Observing that for any finite subset of these sentences in $\{n\cdot c<1|n<\omega\}$ we can find some interpretation of $c$ which works, it follows that each finite subset of $T^*$ has a model, and thus compactness lets us conclude that $T^*$ is consistent.
\end{proof}
\begin{remark}
    This brings us to nonstandard analysis - $c$ is an infinitesimal! In particular, because $\phi$ cannot be axiomatized, we get this interesting model which does not satisfy the Archimedean property.
\end{remark}

We now recall the definition of a well-ordered set.
\begin{boxdefinition}[Well-Ordered Set]
    We say that $(P, \leq)$ is a ``well-order''(ed set) if we have $<$ a linear order such that $\forall S\ssq P$ with $S\neq \emptyset$...\sorry
\end{boxdefinition}
\sorry \sorry \sorry (!) (showed there's an elementary extension of $CD(\oemga, <)$ with $(|M|, <^M)$ is not well-ordered. then talked about periodic and locally finite groups