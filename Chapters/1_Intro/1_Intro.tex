\chapter{An Introduction to Model Theory} \label{Ch1:CH}
\thispagestyle{empty}

Model Theory is one of the two main branches of mathematical logic, alongside Set Theory. We can view Model Theory as a translation of algebra to the world of set theory, or as a \textit{generalisation} of algebra, with many applications to algebra. Indeed, it is possible to view model theory as a more specialised version of category theory, at least in its power to deal with generality. We can prove tremendously deep theorems about algebra, or other fields of maths, purely using model theory.

% We also add a few historical comments that might be interesting to the reader (they were interesting to me). In the 1960s, there was a collection of papers published at two back-to-back maths conferences that together defined a field called ``model theory''. These papers sought to generalise an earlier work by Steinitz from 1910, which, over the course of 200 pages, made substantial progress on field theory, building on the work of the legendary Évariste Galois. In the 70s, more papers on model theory began to appear, generalising the work of Lowenheim and Skolem from the 1910s as well as work by a group based in Warsaw led by Tarski from the late 1920s. After the break of war in 1939 and the murder of several of Tarski's colleagues at the hands of the Nazis, Tarski took refuge in the United States, where he continued his work in UC Berkeley. 

Before we can talk about model theory in any detail, we need to recall a few notions from the world of logic.

\section{A Crash Course on First-Order Logic}

We begin by recalling the notion of a first-order language.

\begin{boxdefinition}[Language]
    A \textbf{language} is a disjoint union
    \begin{align*}
        \Lang = \Funcs \cup \Rels \cup \Consts
    \end{align*}
    of countable sets of symbols, where
    \begin{itemize}
        \item $\Funcs$ is a set of function symbols
        \item $\Rels$ is a set of relation symbols
        \item $\Consts$ is a set of constant symbols
    \end{itemize}
\end{boxdefinition}

Next, we recall the notion of a structure in a language.

\begin{boxdefinition}[Structure]
    Let $\Lang$ be a structure. An \textbf{$\Lang$-structure} is a tuple
    \begin{align*}
        M = \cycl{U; F, R, C}
    \end{align*}
    consisting of a non-empty set $U$ and functions, relations, and constants that live in $\Funcs$, $\Rels$ and $\Consts$ respectively. $U$ known as the \textbf{universe} of $M$, and is denoted by $\abs{M}$. The function, relation and constant symbols of $M$ are denoted $F^M$, $R^M$ and $C^M$ respectively.
\end{boxdefinition}

Any function or relation in a structure has an \textbf{arity}, which is informally the number of arguments it takes. An important fact to note is that arities are not a feature of functions and relations themselves, but of their corresponding \textit{symbols}. In other words, \textbf{arity is a syntactic notion}. Semantically speaking, when we seek an interpretation of a function symbol of some arity $n$, we are forced to limit our search to the set of functions from $U^n$ to $U$.

We now describe the notion of structure-preserving bijections, known as isomorphisms.

\begin{boxdefinition}[Isomorphism]
    Let $\Lang$ be a language and let $M, N$ be $\Lang$-structures. We say that a function $g : \abs{M} \to \abs{N}$ is an \textbf{isomorphism} if
    \begin{enumerate}
        \item $g$ is a bijection
        \item ``$g$ commutes with functions''
        \item ``$g$ commutes with relations''
    \end{enumerate}
    where the double-quotes for the second and third point above refer to the fact that we implicitly require an equality of arities condition before we can talk about composing isomorphisms with multi-ary functions.
\end{boxdefinition}

% Lecture 2

We are now ready to define submodels.

\begin{boxdefinition}[Submodel]
    Let $M, N$ be $\Lang$-structures. We say that $M$ is a submodel of $N$, denoted $M \susbeteq N$, if
    \begin{enumerate}
        \item $\abs{M} \subseteq \abs{N}$.
        \item For all function symbols $F\of{x_1, \ldots, x_n}$, the interpretation in $M$ agrees with the interpretation in $N$.
        \item For all relation symbols $R\of{a_1, \ldots, a_n}$, the interpretation in $M$ agrees with the interpretation in $N$.
        \item For all constant symbols $C$, the interpretation in $M$ agrees with the interpretation in $N$.
    \end{enumerate}
\end{boxdefinition}

In particular, a submodel of a model is also a model. For instance, if $G$ is a group, ie, a model of the group axioms, then any submodel of $G$ is, in fact, a subgroup of $G$, and a group in its own right (in that it again models the group axioms).

We now talk about more syntactic elements of a language.

\begin{boxdefinition}[Terms]
    Let $\Lang$ be a language. $\Term(\Lang)$ is defined to be the minimal set of finite sequences of symbols\footnote{Here, the set of variable symbols is countable, but we might, on occasion, need uncountably many variable symbols} from
    \begin{align*}
        \set{(, ), [, ]} \cup \Consts \cup \Funcs \cup \set{x_1, x_2, x_3, \ldots}
    \end{align*}
    satisfying the following rules:
    \begin{enumerate}
        \item Every constant symbol is a term.
        \item Every variable is a term.
        \item For all $n$-ary functions $f$ and $n$ terms $t_1, \ldots, t_n$, $f\of{t_1, \ldots, t_n}$ is a term.
        \item Every term arises in this way.
    \end{enumerate}
\end{boxdefinition}

% For an ULTRA-FORMAL presentation of the material, look at D. Monk Mathematical Logic, Springer GTM. But Rami's feedback is: an undergraduate cannot follow it because it is too technical, and a graduate student will find that it has too much detail but not much actual content.

In similar fashion, we can define the formulae in a language.

\begin{boxdefinition}[Formulae]
    Let $\Lang$ be a language. $\Fml(\Lang)$ is defined to be the minimal set of finite sequences of symbols from
    \begin{align*}
        \Term\of{L} \cup \set{\land, \lor, \to, \neg, \ldots} \cup \set{=} \cup \set{\forall, \exists}
    \end{align*}
    satisfying the following rules:
    \begin{enumerate}
        \item For all $\tau_1, \tau_2 \in \Term\of{\Lang}$, $\tau_1 = \tau_2$ is a formula.
        \item For all $n$-ary relations $R$ and $n$ terms $t_1, \ldots, t_n$, $R\of{t_1, \ldots, t_n}$ is a formula.
        \item For all connectives $\star$ and formulae $\Phi$ and $\Psi$, $\Phi \star \Psi$ is a formula.
        \item For a quantifier $Q$, variable $x$ and formula $\varphi(x)$, $Qx \parenth{\varphi\of{x}}$ is a formula.
        \item Every formula arises in this way.
    \end{enumerate}
    Formulae consisting only of a single relation symbol (including formulae that only consist of an equality) are called \textbf{atomic formulae}. The atomic formulae of $\Lang$ are denoted $\AFml(\Lang)$.
\end{boxdefinition}

For a formula $\varphi$, denote by $\FV{\varphi}$ the set of free variables of $\varphi$. It is sometimes useful to distinguish those formulae in a language that contain no free variables.

\begin{boxdefinition}
    Define the set of \textbf{satisfiable} formulae of a language $\Lang$ to be
    \begin{align*}
        \Sat\of{\Lang} := \setst{\varphi \in \Fml\of{\Lang}}{\FV{\varphi} = \emptyset}
    \end{align*}
\end{boxdefinition}

Finally, we make the most important definition of this section.

\begin{boxdefinition}[Satisfaction]
    Let $\Lang$ be a language and $M$ an $\Lang$-structure. For all $\varphi \in \Fml\of{\Lang}$, we say that \textbf{$M$ models $\varphi$}, denoted $M \models \varphi$, if \sorry % This is soooo tedious fr
\end{boxdefinition}

We are now ready to state a simple-sounding but rather non-trivial result.

\begin{boxlemma}
    Suppose $M$ and $N$ are both $\Lang$-structures. If $M \subseteq N$, then for all $\tau \in \Term\of{\Lang}$, $\tau^{M}\!\brac{a} = \tau^N\!\brac{a}$, where $a \in \abs{M} \times \cdots \times \abs{M}$ and $\tau^M\!\brac{a}$ and $\tau^N\!\brac{a}$ denote interpretations of $\tau$ in $M$ and $N$ with the variables all being interpreted as the components of $a$.
\end{boxlemma}

We do not prove this result. It is not difficult.

Next is a less trivial result.

\begin{boxlemma}
    If $M \subseteq N$, then $M \models \varphi$ if and only if $N \models \varphi$ for all quantifier-free formulae $\varphi$.
\end{boxlemma}
\section{Another Section}

\lipsum
