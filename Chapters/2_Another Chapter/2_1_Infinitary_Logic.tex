\section{A Word on Infinitary Logic}

We begin by discussing a fundamental type of logic, the development of which involved the likes of Erdos, Tarski, Henken, Chang, Keisler and Morley.

\subsection{The Syntax and Semantics of $L_{\omega_1, \omega}$}

Let's start with a language $L$. We define a new language out of $L$ called $L_{\omega_1, \omega}$, whose syntax and semantics are as follows.

We define $\Fml\of{\Lomo}$ to be a superset of the set of first-order formulae of $L$, closed under first-order operations, containing also all countable conjunctions and disjunctions of formulae in $L$.

As for the semantics of $\Lomo$, for some $n < \omega$, given formulae
\begin{align*}
    \setst{\vp_{i}\of{x_1, \ldots, x_n}}{i < \omega}
\end{align*}
we say that the $\Lomo$-formulae formed by conjunction and disjunction are satisfied by a structure $M$ and $a_1, \ldots, a_n \in M$ as follows:
\begin{align*}
    M \models \bigwedge_{i < \omega} \varphi\!\brac{a_1, \ldots, a_n} &\iff \text{For every } i < \omega \text{, } M \models \varphi_i\!\brac{a_1, \ldots, a_n} \\
    M \models \bigvee_{i < \omega} \varphi\!\brac{a_1, \ldots, a_n} &\iff \text{There exists } i < \omega \text{ such that } M \models \varphi_i\!\brac{a_1, \ldots, a_n}
\end{align*}
%(I must be very tired...apologies)
%maybe more than I :)))
% and for nothing (you see the quality of the past notes :,))))))
%...to be fixed (lmao) - glad the bar is low (and back we go
% No problem (so am I)
% And thank you very very much for live texing whilst I was gone
% The quality is fine tbh
% Like there were errors, but they took all of 2 minutes to fix
% So you're quite alright
% You didn't break the document!! :D
% Nah nah nah bar ain't low

\begin{boxexample} Say we let $L=L_{PA}=\langle +, \cdot, 0, 1\rangle$ denote the language of Peano Arithmetic. While the compactness theorem gives us nonstandard models of Peano Arithmetic, in infinitary logic we have the sentence $\psi=\wedge PA \wedge [\forall x[x=0)\vee(\vee_{n<\omega}(x=\sum_{i=1}^n 1)]]$ - in particular we have that $M\models \psi \Leftrightarrow M\cong (\omega, +, \cdot, 0, 1)$.

Similarly, with the use of similar infinite sentences (using only finitely many variables) we can axiomatize Archimedean fields and periodic groups.
\end{boxexample}

\subsection{New Languages using Infinite Cardinals}

We can actually talk about infinitary logic more generally, where given a language $L$, we construct languages $L_{\lambda^+, \omega}$, with the specialisation $\lambda = \aleph_0$ giving us $\Lomo$. The semantics of a language $\Lo{\lambda^+}$ allow formulae of the form
\begin{align*}
    \bigwedge_{\alpha < \lambda} \vp_{\alpha}\of{x_1, \ldots, a_n}
    \qquad \text{ and } \qquad
    \bigvee_{\alpha < \lambda} \vp_{\alpha}\of{x_1, \ldots, a_n}
\end{align*}
That is, $\Lop{\lambda}$ allows quantification of $\kappa$-many elements for any cardinal $\kappa < \lambda$.

Interestingly, if we consider cardinals $\aleph_0 \leq \lambda < \mu$, we can show that $\Lo{\lambda^+} \subsetneq \Lo{\mu^+, \omega}$.

It was believed by the great Shelah, when he first came to the US in the 1950s, that the future of logic was not model theory but infinitary logic. He defined the notion of an Abstract Elementary Class (AEC), the title of this chapter, which is also the sort of thing Professor Grossberg and his PhD students do research on. The point of AEC is that it defines a concrete category, consisting of objects that are sets with structure and morphisms that are structure-preserving injections, and admitting projective limits and certain other category theoretic properties. But we will not discuss these here.

\subsection{The Intuition of Abstract Elementary Classes}

Fix a language $L$. Let $T$ be a consistent first-order theory in $L$.

First, observe that if $K = \Mod(T)$ is the class of models of this theory, then for any $M, N \in \Mod(T)$, the relation $M \leq_{\Mod(T)} N$, defined to hold if and only if $M$ is an elementary submodel of $N$, defines a partial order on $\Mod(T)$. Moreover, any subclass $K \ssq \Mod(T)$ is also partially ordered by the restriction of this relation, which we denote $\leq_K$.

We begin by defining the concept of a Löwenheim-Skolem Cardinal.

\begin{boxdefinition}[The Löwenheim-Skolem Cardinal]
    For any subclass $K \ssq \Mod(T)$, we define its \textbf{Löwenheim-Skolem Cardinal $\LS(K)$} to be the smallest cardinal $\lambda \geq \aleph_0 + \abs{L}$ such that for all $M \in K$ and $A \ssq \abs{M}$, there is $N \in K$ such that $N \leq_K M$, $\abs{N} \geq A$ and $\abs{\abs{N}} \leq \lambda + \abs{A}$
\end{boxdefinition}

We can use this to define an AEC in the following manner.

\begin{boxdefinition}[Abstract Elementary Class]
    We define an \textbf{Abstract Elementary Class} to be \textit{any} class $K$ of $L$-structures modelling some consistent theory $T$, partially ordered by $\leq_K$ as discussed above, satisfying
    \begin{enumerate}
        \item \underline{Coherence:} if $M_1, M_2, M_3 \in K$, and
        \begin{align*}
            M_1 \leq_K M_3
            \qquad \text{ and } \qquad
            M_2 \leq_K M_3
            \qquad \text{ and } \qquad
            M_1 \ssq M_2
        \end{align*}
        then $M_1 \leq_K M_2$.

        \item \underline{Closure under Isomorphisms:} for all $M \in K$, if $N$ is any $L$-structure such that $M \cong N$, then $N \in K$.

        \item \underline{The Tarski-Vaught Chain Axioms:}
        \begin{enumerate}
            \item For all ordinals $\alpha$ and for all sequences $\setst{M_j}{j < \alpha} \ssq K$, if $i < j$ then $M_i \leq_K M_j$. Moreover,
            \begin{align*}
                M^{*} = \bigcup_{i < \alpha} M_i
            \end{align*}
            also lies in $K$, and for all $i < \alpha$, $M_i \leq M^*$.

            \item If, in addition, we have $N \in K$ such that $\forall i < \alpha$, $M_i \leq_K N$, then $M^* \leq_K N$.
        \end{enumerate}

        \item \underline{The Löwenheim-Skolem Axiom:} \sorry % Something in terms of the LS Cardinal
    \end{enumerate}
\end{boxdefinition}

We also define what it means for a poset to be directed.

\begin{boxdefinition}[Directed Poset]
    Let $\parenth{I, \leq_{I}$ be a poset. We say it is \textbf{directed} if for all $i, j \in I$, there exists $k \in I$ such that $i \leq k$ and $j \leq k$.
\end{boxdefinition}

\begin{boxlemma}
    Suppose $L$ is a countable language. Let $M$ be an uncountable structure with cardinality $\lambda$. Then, there exists some chain $\setst{M_i}{i < \lambda}$ such that
    \begin{enumerate}
        \item For all $i_1, i_2 < \lambda$, $i_1 < i_2 \implies M_{i_1} \leq_K M_{i_2}$
        \item For all $i < \lambda$, $\abs{\abs{M_i}} < \lambda$
        \item $M = \bigcup_{i < \lambda} M_i$
    \end{enumerate}
\end{boxlemma}
\begin{proof}
    \sorry
\end{proof}

We can now state an important theorem about AECs. Note that the first point, while similar, is subtly different (if not very significantly so) from the Tarski-Vaught Axioms because it involves posets rather than ordinals.

\begin{boxtheorem}
    Let $K$ be an AEC and let $\parenth{I, \leq}$ a directed poset. Then,
    \begin{enumerate}[label = (\Alph*)]
        \item For all families $\setst{M_i}{i \in I} \sse K$ such that if $i \leq j \implies M_i \leq_K M_j$, if we define
        \begin{align*}
            M^* = \bigcup_{i \in I} M_i
        \end{align*}
        then $M^* \in K$ and $\forall j \in I$, $M_j \leq_K M^*$.

        \item If, in addition, there is some $N \in K$ such that $\forall i \in I$, $M_i \leq_K N$, then $M^* \leq_K N$.
    \end{enumerate}
    Categorically speaking, the first point says something about the existence of limits.
\end{boxtheorem}
\begin{proof}
    We argue by induction on $\abs{I} = \aleph_{\alpha}$.

    \begin{enumerate}
        \item \underline{$\alpha = 0$.} In this case, enumerate $\setst{a_n}{n < \omega} = I$. By induction on $n$, using the directedness property of the poset $I$, define $\setst{b_n}{n < \omega} \ssq I$ such that $b_{n+1} \geq b_n$ and $b_{n+1} \geq a_n$. Notice that
        \begin{align*}
            M^* = \bigcup_{i \in I} M_i = \bigcup_{n < \omega} M_{b_n}
        \end{align*}
        a fact that is true because of condition (A). Apply the Tarski-Vaught Chain Axioms to $J = \setst{b_n}{n < \omega}$ to conclude. % ???

        \item \underline{$\alpha > 0$.} Let $\lambda$ be the Löwenheim-Skolem Cardinal. By the lemma (\sorry - see HW 2), there is an elementary chain $\setst{I_j \leq I}{j < \lambda}$ such that $\abs{I_j} < \lambda$ for all $j$ and $\bigcup_{j} I_j = I$.
        
        Since $I$ is directed,
        \begin{align*}
            \parenth{I, \leq} \models \forall x \forall y \exists z \brac{z \geq y \land z \geq x}
        \end{align*}
        This implies that $\parenth{I_j, \leq}$ is also directed. Apply the induction hypothesis to $\setst{M_i}{i \in I_j}$ for all $j$. Let
        \begin{align*}
            M_j^* = \bigcup_{i \in I_j}
        \end{align*}
        Then, since $j_1 < j_2 \implies I_{j_1} \ssq I_{j_2}$, we have that $M_{j_1}^* = M_{j_2}^*$. By (A), for all $i \in I_{j_l}$, $M_i \leq_K M_{j_l}$. Applying (B) to $\setst{M_i}{i \in I_1}$ to obtain \sorry
    \end{enumerate}
\end{proof}