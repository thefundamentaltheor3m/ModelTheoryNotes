\section{The Erd\H{o}s-Rado Theorem}

As usual (comme d'habitude), fix a language $L$.

\subsection{Une Perspective Galoisienne}

We begin by defining the notion of a type\footnote{No, this is NOT a computer sciency thing!}.
%:) (I meant ':(' of course)
%hmmmmm... there's an 'Erdos-Ko-Rado' theorem as well 
% HAha
\begin{boxdefinition}[Type]
    Let $M$ be an $L$-structure. Fix $A \subseteq \abs{M}$.
    % What is this definition
    Let $\ol{b} = \parenth{b_1, \ldots, b_n} \in \abs{M}^n$. We define
    \begin{align*}
        \tp\of{\quotient{\ol{b}}{A}, M}
        =
        \setst{\varphi\of{\ol{x}; \ol{a}}}{\vp\of{\ol{x}; \ol{y}} \in \Fml(L) \text{ and } \ol{a} \in A \text{ with } M \models \vp\!\brac{\ol{b}, \ol{a}}}
    \end{align*}
    That is, we define $\tp\of{\quotient{\ol{b}}{A}, M}$ to be the set of all $L$-formulae with parameters $\ol{a} \in A$ and free variables $\ol{x}$ such that when the $\ol{x}$ are interpreted as $\ol{b}$ in $M$, $\vp$ is satisfied.
\end{boxdefinition}

We state a preliminary result about types.

\begin{boxlemma}
   Suppose $M_1 \leq M_2$ are $L$-structures and $A \ssq \abs{M_1}$, $\ol{b} \in \abs{M_1}$. Then
   \begin{align*}
       \tp\of{\quotient{\ol{b}}{A}, M_1} = 
       \tp\of{\quotient{\ol{b}}{A}, M_2}
   \end{align*}
\end{boxlemma}
\begin{proof}
    The idea is to exploit the elementarity of $M_1 \leq M_2$. \sorry
\end{proof}

We will use the theory of types to take an approach reminiscent of Galois Theory to prove the famous \textbf{Erd\H{o}s-Rado Theorem}, which can be thought of as a version of the pigeonhole principle for uncountable cardinals.